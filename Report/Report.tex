\documentclass[a4paper, openright, twosides]{book}
\usepackage{geometry}
\geometry{hmargin={3cm,3cm},vmargin={3cm,3cm}}
\usepackage[english,italian]{babel}
\usepackage[T1]{fontenc}
\usepackage[utf8]{inputenc}
\usepackage{fancyhdr}
\usepackage{float}
\usepackage{graphicx}
\usepackage{wrapfig}
\usepackage{siunitx} %per scrivere il simbolo °
\usepackage{verbatim} %per i commenti1
\usepackage{subfig}
\usepackage{amsmath}
\usepackage[linesnumbered, ruled, vlined]{algorithm2e}
\usepackage{algorithmic}
\setcounter{secnumdepth}{3}
\setcounter{tocdepth}{6}
\usepackage{multirow}
\newcommand{\minitab}[2][l]{\begin{tabular}#1 #2\end{tabular}}
\usepackage{rotating}
\usepackage{xfrac}
\usepackage{enumitem}
\usepackage{amsmath}
\usepackage{showlabels}
\usepackage{cases}
\usepackage{multicol}

%CSV
\usepackage{array,booktabs,longtable,csvsimple}

\csvstyle{mystyle}{
    tabular=ccc,
    table head=\toprule,
    table foot=\bottomrule,
    no head,
    late after line=\\,
    late after first line=\\\midrule,
    }

\DeclareMathOperator*{\argmax}{arg\,max}
\DeclareMathOperator*{\argmin}{arg\,min}

%\usepackage{booktabs,array}
%\usepackage{tikz}

%\usepackage{tabularx}

%\usepackage{chngcntr}
%\counterwithin{table}{section}

%------------------------------ colors
\usepackage[usenames,dvipsnames,table]{xcolor} % use colors on table and more
\definecolor{333}{RGB}{51, 51, 51} % define custom color
\definecolor{background}{RGB}{255, 254, 213}
\definecolor{comment}{RGB}{17,167,5}
\definecolor{keyword}{RGB}{195,47,8}
\definecolor{string}{RGB}{142,195,0}
\definecolor{number}{RGB}{90,84,84}
\definecolor{identifier}{RGB}{0,90,201}

%------------------------------ source code
\usepackage{listings}

\lstset{
  basicstyle=\footnotesize\sffamily,
  commentstyle=\itshape\color{gray},
  captionpos=b,
  frame=shadowbox,
  language=HTML,
  rulesepcolor=\color{333},
  tabsize=2
}

\lstdefinestyle{code}{
  backgroundcolor=\color{background},
  basicstyle=\footnotesize\sffamily,
  commentstyle=\color{comment},
  frame=L,
  identifierstyle=\color{identifier},
  keywordstyle=\color{keyword},
  numbers=left,
  numbersep=10pt,
  numberstyle=\tiny\color{number},
  stringstyle=\color{string},
  showstringspaces=false,  
  stepnumber=1,
  tabsize=2
}


%------------------------------ define Abstract environment, missing in the 'book' class
\newenvironment{abstract}{\cleardoublepage \null \vfill \begin{center}\bfseries\abstractname \end{center}}{\vfill\null}
\addto\captionsenglish{\renewcommand*\abstractname{Abstract}} % change Abstract title

%------------------------------ active url
\usepackage{url}
\renewcommand{\UrlFont}{\color{black}\small\ttfamily}
\usepackage[colorlinks=true, linkcolor=black, citecolor=black, urlcolor=black]{hyperref} % active ref
%------------------------------ macros
\newcommand{\sectionname}{Section} % define Section ref
\newcommand{\subsectionname}{Sub-section} % define Sub-section ref
\renewcommand*\arraystretch{1.4} % tables padding

%acronimi
\usepackage[printonlyused]{acronym}

\begin{document}
\frontmatter

\begin{titlepage} %------------------------------ TITLE PAGE
\begin{center}

\hspace{0.5cm}
\begin{minipage}{.20\textwidth}
  \includegraphics[height=2.5cm]{Images/UNIPD}
\end{minipage}\begin{minipage}{.90\textwidth}
  \begin{table}[H]
  \begin{tabular}{l}
  \scshape{\Large{\bfseries{Università degli Studi di Padova}}} \\
  \hline \\
  \scshape{\Large{Facoltà di Ingegneria}} \\
  \end{tabular}
  \end{table}
\end{minipage}

\vspace{1cm}
\emph{\Large{Corso~di~Laurea~Magistrale~in~Ingegneria~Informatica}} \\
\vspace{3cm}
\scshape{\Large{Tesina di Ricerca Operativa 2}} \\
\end{center}

\vspace{1cm}
\begin{center}
\scshape{\Large{\bfseries{Travelling Salesman}}}\\
\scshape{\Large{\bfseries{Problem}}}
\end{center}

\vspace{3.5cm}

\begin{center}
\emph{Autori}
\vspace{0.2cm}
\begin{table}[h]
\centering
\begin{tabular}{rl}
\vspace{0.2cm}
{Raffaele Di Nardo Di Maio} & {1204879}\\
{Cristina Fabris} & {1205722}\\
\end{tabular}
\end{table}

\end{center}

\vfill
\begin{center}
\hspace{-0.2cm}
\line(1, 0){360}\\
\textsc{Anno Accademico 2019-2020}
\end{center}
\end{titlepage}

%\begin{comment}
\begingroup %------------------------------ CONTENTS
  \makeatletter
  \let\ps@plain\ps@empty
  \makeatother
  \tableofcontents  
  \clearpage
\endgroup
%\end{comment}

\SetAlgorithmName{Algoritmo}{Algoritmo}

\begin{comment}
\makeatletter %REMOVE Chapter Title in each chapter but not in the index
\def\@makechapterhead#1{%
  \vspace*{50\p@}%
  {\parindent \z@ \raggedright \normalfont
    \vskip 20\p@
    \interlinepenalty\@M
    \Huge \bfseries #1\par\nobreak
    \vskip 40\p@
  }}
\makeatother
\end{comment}

\mainmatter
\chapter{Introduzione}\label{intro}
La seguente trattazione analizza il Problema del Commesso Viaggiatore (Travelling Salesman Problem, TSP), che consiste nell'individuare un circuito hamiltoniano di costo minimo in un assegnato grafo orientato G=(V,A)\cite{TSP}. La formulazione matematica di tale problema è la seguente:
$$
x_{ij}=
\begin{cases}
1 & se\;l'arco\;(i,j)\in A\;viene\;scelto\;nel\;circuito\;ottimo\\
0 & altrimenti\\
\end{cases}
$$
\begin{align}
& min \underset{(i,j)\in A}\sum{c_{ij}\;x_{ij}} \\\notag \\
& \underset{(i,j)\in \delta^{-}(j)}\sum{\;x_{ij}} = 1 & \forall\;j\in V \\\notag \\
& \underset{(i,j)\in \delta^{+}(i)}\sum{\;x_{ij}} = 1 & \forall\;i\in V \\\notag \\
& \underset{(i,j)\in \delta^{+}(S)}\sum{\;x_{ij}} \geq 1 & S \subset V\; :\; 1\in S\\\notag \\
& x_{ij}\geq 0\;\; intero & (i,j)\in A.\\\notag
\end{align}
Tuttavia le soluzioni algoritmiche presentante risolvono una sua variante, detta simmetrica, che viene applicata ad un grafo completo non orientato G=(V,E).\\ Di seguito viene riportata la formulazione matematica di tale versione:\\
\begin{align}
& min \underset{e\in E}\sum{c_e\;x_e}\\
& \underset{e\in \delta(v)}\sum{\;x_e} = 2 & \forall\;v\in V \\
& \underset{e\in E(S)}\sum{\;x_e} \leq \vert S\vert - 1 & \forall\;S\underset{\neq} \subset V: \vert S\vert\geq 3.\\\notag
\end{align}
A livello commerciale esistono diverse tipologie di risolutori di problemi di programmazione lineare intera, basati sul Branch \& Bound. I più conosciuti in circolazione sono i seguenti:
\begin{itemize}
\item{\textbf{IBM ILOG CPLEX Optimization Studio}\cite{ILOG}\\
è un soluzione analitica, sviluppata dall'IBM e gratuita a livello accademico.}
\item{\textbf{FICO® Xpress Optimization}\cite{FICO}\\
è stato prodotto dalla Fair Isaac Corporation(FICO) ed è costituito da 4 componenti principali: FICO® Xpress Insight, FICO® Xpress Executor, FICO® Xpress Solver e FICO® Xpress Workbench. Questa soluzione è disponibile gratuitamente solo nella versione Community, in cui però vengono applicate restrizioni sul numero di righe e colonne del tableau, di token non lineari e di funzioni dell'utente.}
\item{\textbf{Gurobi}\cite{GUROBI}\\
è una soluzione, sviluppata dalla Gurobi Optimization, che viene rilasciata anche con una versione accademica.}
\item{\textbf{COIN Branch and Cut solver (CBC)}\cite{CBC}\\
è un risolutore MIP(mixed-integer program) open-source scritto in C++ e sviluppato dalla Computational Infrastructure for Operations Researc (COIN).}
\end{itemize}
Nel Capitolo \ref{CPLEX} vengono riportate diverse soluzioni math-euristiche e non per il problema del Commesso Viaggiatore, che fanno uso di ILOG CPLEX.\\
In commercio, il più noto ed efficiente software per la risoluzione del TSP è Concorde, sviluppato in ANSI C e disponibile per l'uso in ambito accademico\cite{concorde}.\\
Nel Capitolo \ref{HEURISTIC} vengono analizzati gli algoritmi euristici, sviluppati senza far uso di ILOG CPLEX.\\
Nel Capitolo \ref{PERF_PROF} vengono invece riportati i confronti, a livello temporale e di costo, delle soluzioni ottenute con i differenti algoritmi enunciati.\\
Nell'Appendice \ref{TSPlib}, \ref{CPLEX_func}, \ref{gnuplot}, \ref{perf_profile_py} vengono descritte rispettivamente la tipologia di istanze utilizzate, la documentazione utilizzata ed il funzionamento di CPLEX, il programma GNUPLOT utilizzato nella stampa delle soluzioni e il programma perfprof.py usato per creare i performance profile del Capitolo \ref{PERF_PROF}.\\
Tutti i tempi di esecuzione e i costi delle soluzioni, ottenuti mediante la fase di testing, sono consultabili invece nelle tabelle riportate nell'Appendice \ref{results}. Tutte le soluzioni descritte sono state implementate in linguaggio C e testate sul sistema operativo Windows 10 con Visual Studio, ed i sorgenti sono disponibili online\footnote{\url{https://github.com/RaffaDNDM/Operational-Research-2}}.
\chapter{Risoluzione del problema tramite CPLEX}\label{CPLEX}  

\section{Modelli compatti}
I modelli compatti del Travelling Salesman Problem, sono formulazioni il cui numero di variabili e di vincoli è polinomiale nella taglia dell'istanza. In particolare, in quelle analizzate in seguito, sono entrambi $O(n^2)$, con \textit{n = numero di nodi}.\\
I modelli compatti sono però applicabili solo a grafi orientati. Per poterli sfruttare per la risoluzione del TSP simmetrico, è necessario per ogni ramo dell'istanza $(i,j)$, inserire nel modello i corrispondenti rami orientati in entrambe le direzioni $(i,j)$ e $(j,i)$. Questo comporta un significativo rallentamento nella computazione della soluzione, in quanto l'algoritmo, ogni volta che scarta un ramo $(i,j)$ dalla soluzione ottima, verifica se il corrispondente $(j,i)$ potrebbe invece appartenerle. Questo non può però essere possibile, essendo i due rami in realtà lo stesso nella nostra istanza iniziale.
\subsection{Formulazione sequenziale}
Miller, Tucker e Zemlin, nella loro formulazione del modello, hanno introdotto una nuova variabile $u_i$ per ogni nodo \textit{i} e imposto che, nella soluzione ottima, il suo valore  rispettasse dei nuovi vincoli. Questi servivano a garantire che venisse seguito un ordine di percorrenza di tutti i nodi. In questo modo hanno eliminato la creazione di sub-tour, mantenendo il numero di vincoli e di variabili polinomiale. Nello specifico il loro modello è così strutturato:
\begin{align}
& min \underset{i\in V}\sum{}\underset{j\in V}\sum{c_{i,j}\; x_{i,j}} \\ \notag \\
& \underset{i\in V}\sum{x_{ih}}\; =\; 1 & \forall\; h\in V \\ \notag \\
& \underset{j\in V}\sum{x_{hj}}\; =\; 1 & \forall\; h\in V \\ \notag \\
& u_i-u_j+n\; x_{i,j}\;\leq\; n-1 & \forall\; i,j\in V-\{ 1\} , i\neq j \\ \notag \\
& 0\leq u_i\;\leq\; n-2 & \forall\; i\in V-\{ 1\} \\ \notag 
\end{align}
Esistono due diversi modi per implementare questo modello sfruttando le funzioni di CPLEX.\\
Nel primo i nuovi vincoli vengono aggiunti come visto in precedenza. In questo modo, durante la fase di preprocessamento, il programma è già a conoscenza di tutti i vincoli che dovrà rispettare la soluzione ottima. Ciò gli permette di migliorare i coefficienti presenti, prima ancora di iniziare la computazione dell'ottimo.\\
Il secondo metodo, invece, sfrutta l'inserimento nel modello di vincoli detti "lazy constraints". Questi non sono noti al programma dall'inizio, ma vengono inseriti all'interno di un pool di vincoli. Nel momento in cui viene calcolata una soluzione, CPLEX verifica che vengano rispettati tutti i vincoli presenti nel pool. Se ne trova uno violato lo aggiunge al modello e ripete la computazione. Questo approccio permette, per risolvere lo stesso problema, di eseguire calcoli su un modello più piccolo, ma può aumentare i tempi di computazione non fornendo a CPLEX tutte le informazioni dall'inizio.   

\subsection{Formulazione basata sul flusso}
Nella formulazione di Gavish e Graves, per impedire la formazione di sub-tour all'interno della soluzione ottima, viene introdotto un nuovo vincolo per ogni ramo del grafo. Questo permette di regolare il flusso $y_{i,j}$, con $i\neq j$,  che lo attraversa.Inoltre è stato necessario aggiungere anche dei vincoli, detti "vincoli di accoppiamento", che collegassero i flussi alle variabili $x_{i,j}$ . Il loro modello è quindi così strutturato:
\begin{align}
& min\underset{i\in V}\sum{\underset{j\in V}\sum{c_{i,j}\; x_{i,j}}} \\ \notag \\
& \underset{i\in V}\sum{x_{ih}}\;=\;1 & \forall\;h\in V \\ \notag\\
& \underset{j\in V}\sum{x_{hj}}\;=\;1 & \forall\;h\in V \\ \notag\\
& \underset{j\in V}\sum{y_{1,j}}\;=\;1\\ \notag\\
& \underset{j\in V}\sum{y_{h,j}}\;=\;\underset{i\in V}\sum{y_{i,h}}\;-1 & \forall\;h\in V-\{1\}\\ \notag\\
& y_{i,j}\leq\;(n-1)\;x_{i,j} & \forall\;i,j\in V, i\neq j\\ \notag
\end{align}
La soluzione di questo modello risulta però essere lontana dalla convex hull. Per migliorarla è possibile sostituire il vincolo \textit{(3.11)} con \\
\begin{center}
$y_{i,j}\leq\;(n-2)\;x_{i,j} \;\;\;\;\;\forall\; i\neq \; j$\\
\end{center}
mentre per gli altri valori di $i$ e $j$ è necessario lasciare i vincoli originali. 
Per evitare che la soluzione ottima contenga sia l'arco $x_{i,j}$ che $x_{j,i}$, che nella nostra istanza iniziale corrispondono allo stesso arco, viene anche aggiunto il seguente vincolo:
\begin{center}
$x_{i,j}+x_{j,i}\leq 1\;\;\;\; \forall i,j \in V$ con $i < j$
\end{center}

\section{Loop}
\subsection{Formulazione di Benders}
Negli anni '60, Jacques F. Benders sviluppò un approccio generale, applicabile a qualsiasi problema di programmazione lineare, per ridurre il numero esponenziale di alcuni vincoli specifici inseriti nel modello.\\
Utilizzando questo metodo, il modello viene scritto senza quei vincoli e poi questi verranno aggiunti in seguito durante la risoluzione del problema. Nel caso in cui la soluzione ottima, calcolata a partire da questo modello, non rispetti un vincolo di quelli rimossi, questo viene aggiunto al modello. \\
Nella seguente parte, viene riportata l'applicazione specifica di loop al problema TSP.
I vincoli di Subtour Elimination sono in numero esponenziale e sono:\\
\begin{align}
&\underset{e\in E(S)}\sum{x_{e}} \leq \mid S\mid - 1\;\forall\;S\underset{\neq}{\subset}V\; : \; \mid S\mid\geq 2
\end{align}
o equivalentemente:
\begin{align}
&\underset{e\in \delta(S)}\sum{x_{e}}\geq 2\;\forall\;S\underset{\neq}{\subset}V\; : \; \mid S\mid\geq 2
\end{align}
Viene definito un nuovo modello per il problema del commesso viaggiatore simmetrico, in cui vengono rimossi tali vincoli, aggiungendo così la possibilità di avere dei subtour nella soluzione finale.\\
Viene risolto il problema e nel caso in cui ci sia più di una componente connessa, viene aggiunto al modello un vincolo di subtour elimination per ogni 
ciclo generato.\\
\begin{algorithm}[H]
\DontPrintSemicolon
\KwIn {$\mathtt{model}$= Modello TSP simmetrico senza vincoli di Subtour Elimination\newline}
\KwOut {$\mathtt{x}$= soluzione intera senza subtour}
\BlankLine 
 $\mathtt{x} \gets$ solve(model)\;
 $\mathtt{ncomps} \gets$ comps(x)\;
 \BlankLine 
 \While{$\mathtt{ncomps} \geq 2$}{
  Aggiungi $\underset{e\in \delta(S_k)}\sum{x_{e}}\leq \mid S_k\mid - 1\;\forall$ componente connessa $S_k$\;
  \BlankLine  
  \If{$\mathtt{ncomps} \geq 2$}{
    \BlankLine
    $\mathtt{x} \gets$ solve(model)\;
    $\mathtt{ncomps} \gets$ comps(x)\;
  }
 }
 \caption{LOOP}
\end{algorithm}

All'aumentare del numero di vincoli, il costo della soluzione ottenuta da CPLEX peggiora o resta identica a quella elaborata all'iterazione precedente del metodo loop.\\
Il numero di iterazioni che vengono effettuate dall'algoritmo non è conosciuto e potrebbe essere anche molto elevato. Nel caso peggiore vengono inseriti tutti i vincoli di Subtour elimination, ovvero un numero esponenziale di disequazioni, soprattutto con istanze clusterizzate.\\
Inoltre il problema principale di questo algoritmo è la generazione, ad ogni iterazione, di un albero completo di branching, eliminando quello precedentemente sviluppato.\\
In passato, con le versioni del MIP solver di CPLEX degli anni '60, questa operazione era molto onerosa mentre attualmente il metodo loop garantisce la risoluzione, anche di istanze molto grandi, in tempi ragionevoli.
Questo non accade invece per il Branch \& Bound in quanto vengono aggiunte nuove ramificazioni all'albero già esistente.\\
L'introduzione di nuovi vincoli di Subtour Elimination, solo nel momento in cui si presenta una loro violazione, permette di ridurre la dimensione del modello ma riduce l'attività di pre-processamento svolta da CPLEX prima di cominciare la risoluzione del problema. Nella fase di pre-processing infatti, vengono applicati algoritmi euristici e cambiamenti dei coefficienti nel modello, in base ai vincoli inseriti.\\
L'algoritmo può essere modificato svolgendo prima il metodo loop con l'aggiunta di parametri differenti da quelli utilizzati di default del risolutore CPLEX. In seguito viene effettuato nuovamente l'algoritmo di Benders ma questa volta nella sua versione esatta, in modo da migliorare la soluzione meta-euristica trovata nella prima parte. \\
Quest'ottimizzazione è basata sul fatto che CPLEX salvi alcune soluzioni, ottenute in precedenza dal risolutore sullo stesso modello, e le sfrutti come bound nel nuovo modello. Per questo motivo, alcune delle soluzioni metaeuristiche ottenute nella prima fase vengono sfruttate come bound nella seconda.

\subsection{Formulazione con callback}
Un possibile miglioramento dell'algoritmo proposto da Benders, in termini di velocità della computazione, è il seguente.\\
Come precedentemente descritto, come prima cosa CPLEX effettua un preprocessamento in cui semplifica il modello, riducendo il termine noto e accorpando tra loro diverse variabili. Terminata questa operazione inizia ad eseguire la fase di Branch \& Cut. Ogni volta che calcola una nuova soluzione $x^*$, prima di dichiarare se è l'ottimo o di scartarla e proseguire a sviluppare i successivi rami dell'albero decisionale, applica dei tagli e degli algoritmi euristici per aggiornarla(vedi Figura \ref{Albero_decisionale}).
\begin{figure}[h] 
\begin{center} 
  \includegraphics[scale=0.7]{Images/albero_decisionale}\\ 
  \caption{\footnotesize{Albero decisionale del Branch and Cut}}
  \label{Albero_decisionale} 
\end{center} 
\end{figure}
\\Nello sviluppo di ogni ramo l'upper bound sarà dato dagli algoritmi euristici utilizzati, mentre il lower bound dal rilassamento del problema.\\ 
Per poter velocizzare l'approccio proposto da Benders, è possibile personalizzare questa fase e scegliere quali tagli far applicare a CPLEX. Nel nostro caso, questi vengono utilizzati per eliminare l'eventuale presenza di subtour nella soluzione calcolata. Per fare ciò vengono sfruttate particolari funzioni fornite da CPLEX, dette \textit{callback}. Queste sono state lasciate volutamente vuote dai creatori della libreria, affinché l'utente possa implementarne all'interno il suo specifico codice. In particolare, le funzioni utilizzate sono callback necessarie ad aggiungere lazy constraints al modello e per questo dette \textit{lazy constraints callback}. La callback implementata viene chiamata solo al momento di aggiornare l'incumbent e se necessario aggiunge al modello i vincoli violati. Verrà quindi invocata più frequentemente all'inizio del calcolo della soluzione del problema, e meno nelle iterazioni successive. Questo poichè essendoci in partenza meno vincoli, sarà più facile per la soluzione soddisfarli tutti. A differenza dei \textit{lazy constraints}, con l'utilizzo delle \textit{lazy callback} i vincoli non sono costantemente presenti in un pool, ma vengono generati "al volo" al momento necessario.  Quest'operazione velocizza notevolmente il calcolo della soluzione ottima, in quanto permette a CPLEX di non dover calcolare nuovamente l'albero decisionale dalla radice, ma di proseguirne lo sviluppo aggiungendo nuovi rami. In particolari casi, però, CPLEX può ritenere più conveniente distruggere tutto l'albero decisionale fin'ora calcolato e ricominciare dalla radice. Questo può avvenire in qualunque punto dell'elaborazione della soluzione ottima.\\
Attraverso l'utilizzo delle callbacks è possibile accedere a molti dati interni all'elaborazione di CPLEX. Particolari procedure vengono quindi automaticamente disattivate, affinché l'utente non possa venirne a conoscenza. Per evitare questo è possibile installare le callbacks con una modalità leggermente diversa, attraverso funzioni dette \textit{general}.
\section{Algoritmi Math-Heuristic}
Gli algoritmi euristici sono progettati per risolvere istanze del problema in tempi significativamente più brevi rispetto agli algoritmi esatti. Di conseguenza, però, al termine della computazione non garantiscono di ottenere una soluzione ottima, ma solo una sua buona approssimazione ammissibile. Gli algoritmi Math-Heuristic sfruttano l'approccio degli euristici, assieme all'utilizzo di un maggior numero di vincoli nel modello, vincoli basati su procedimenti matematici. L'algoritmo che maggiormente rappresenta questo metodo è il Soft Fixing (vedi sottosezione \ref{soft fixing}). \\
Durante la computazione della soluzione CPLEX utilizza diversi algoritmi euristici, grazie alla variazione di alcuni parametri a loro associati è possibile variare la frequenza o il tempo a loro dedicato. 
\subsection{Hard Fixing}\label{hard fixing}
Un primo algoritmo euristico di semplice implementazione si basa sull'impostazione di una deadline da parte dell'utente ed è composto dalle seguenti fasi:
\begin{enumerate}
\item{Impostazione di un time limit per la computazione della soluzione;}
\item{Calcolo della soluzione;}
\item{Selezione, in maniera randomica, di un sottoinsieme di rami appartenenti alla soluzione ottima (Figura \ref{selezione_rami}). 
Il numero di questi sarà dato da una percentuale fissata del totale. I rami appartenenti alla selezione vengono fissati di modo tale che, in una successiva computazione del problema, appartengano alla soluzione restituita;}
\end{enumerate} 
Questi passaggi vengono eseguiti in maniera ciclica per un numero fissato di iterazioni. In questo modo, ad ogni computazione della soluzione, CPLEX dovrà risolvere un problema più semplice di quello originale, essendo molte variabile del modello già selezionate nella \textbf{fase 3} dell'iterazione precedente.\\
Il time limit nominato \textbf{fase 1} è dato da una frazione della deadline complessiva e dipende dal numero di valori di diverse percentuali che si desidera utilizzare. All'ultima iterazione il time limit viene ricalcolato in base al tempo rimanente affinchè venga sfruttata tutta la deadline disponibile. Ad ogni computazione la soluzione potrà essere solo migliore o uguale alla precedente (nel caso peggiore).\\
La percentuale scelta del numero di rami da fissare può variare ad ogni iterazione. Generalmente si cerca di avere una percentuale alta nelle prime iterazione, in cui la soluzione non è ancora stata raffinata, e di abbassarla man mano che si procede con l'algoritmo. In questo modo nelle ultime computazione CPLEX avrà un maggior numero di gradi di libertà per trovare la soluzione che più si avvicina all'ottimo. Poichè i rami selezionati nella \textbf{fase 3} sono scelti in maniera casuale, non si corre il rischio di entrare in un ciclo infinito, in cui viene risolto ogni volta la stessa istanza con le stesse variabili fissate. Particolare attenzione deve essere posta al fatto di lasciare nell'insieme dei rami scelti randomicamente solo quelli selezionate all'iterazione immediatamente precedente.\\
Nella nostra implementazione abbiamo scelto di utilizzare come percentuale i seguenti valori $\lbrace$ 90, 75, 50, 25, 10, 0 $\rbrace$.\\
Di seguito viene inoltre riportato lo pseudocodice dell'algoritmo appena descritto.
\begin{figure}[h] 
\begin{center} 
  \includegraphics[scale=0.38]{Images/x_best} 
  \caption{\footnotesize{Selezione rami}}
  \label{selezione_rami} 
\end{center} 
\end{figure}

\begin{algorithm}[H]
\DontPrintSemicolon
\KwIn {$\mathtt{model}$= Modello TSP simmetrico senza vincoli di Subtour Elimination \newline
$\mathtt{deadline}$= time limit complessivo dell'algoritmo\newline
$\mathtt{percentage}$= array con i valori delle percentuali di fissagio degli archi\newline
$\mathtt{num_nodi}$= numero di nodi dell'istanza tsp\newline}
\KwOut {$\mathtt{x}$= soluzione intera senza subtour}
\BlankLine
$\mathtt{n \gets}0$\;
\BlankLine
\While{$expired\_time < deadline$}{
 \BlankLine\BlankLine
 $\mathtt{setTimeLimit()}$\;
 $\mathtt{x \gets solve(}model\mathtt{)}$\;
 \BlankLine\BlankLine
 \For{$\mathtt{j \gets}0$ \KwTo $num\_nodi-1$}{
   $k \gets random(0,1)$\;
   \BlankLine\BlankLine
   \If{$  100*k \leq percentage[n\;mod\;leght(percentage)]  $}{
     \BlankLine
     $Aggiungi \;\;\mathtt{x\_best[j]}\;\;to\;\;S\;where\;S=\{edges\;to\;fix\}$
   }
   \BlankLine
}
\BlankLine
\ForAll{$x_{i,j} \in S$}{
$\mathtt{x_{i,j} \gets}1$\;
\BlankLine
}
\BlankLine
$\mathtt{n \gets}n+1$\;
\BlankLine
}
\caption{Hard Fixing}
\end{algorithm}

\subsection{Soft Fixing}\label{soft fixing}
Il metodo seguente fa utilizzo di vincoli aggiuntivi, detti \textbf{Local Branching} e che ha dato il via alla sviluppo della \textbf{Math-Heuristic}, approccio di programmazione matematica (es. tramite CPLEX) unita all'algoritmica euristica\cite{local_branching}.\\
L'approccio utilizzato è simile a quello dell'Hard Fixing, ma la scelta delle variabili da imporre a 1 non viene fatta in maniera randomica ma viene lasciata a CPLEX.\\
Partendo da una soluzione intera ammissibile del TSP $x^H$, viene aggiunto un vincolo sui lati con valore 1 in $x^H$:\\
$$\underset{e\in E\; : \; {x_e}^{H}=1}\sum{x_e}\;\geq\; 0.9\;n$$
dove la sommatoria indica il numero di variabili che vengono preservate a 1 rispetto alla soluzione $x^H$ e \textbf{n} indica il numero di archi selezionati, pari al numero di nodi + 1.\\
In questo caso, il vincolo permetterà a CPLEX di fissare il 90\% dei rami scelti in $x^H$ e avere il 10\% di libertà. Per questo motivo, CPLEX riduce il numero di archi su cui lavorare, in quanto molti nodi hanno già un numero di archi selezionati pari a 2.\\
Un modo alternativo di scrivere lo stesso vincolo è il seguente:
$$\underset{e\in E\; : \; {x_e}^{H}=1}\sum{x_e}\;\geq\; n-k$$
dove k=2,...,20 e rappresenta i gradi di libertà di CPLEX nel raggiungere la nuova soluzione.\\
Ad ogni iterazione viene aggiunto un nuovo local branching, basato sull'attuale soluzione restituita da CPLEX, e rimosso il vincolo aggiunto nell'iterazione precedente.\\
L'unico aspetto negativo di questo metodo riguarda un mancato miglioramento della soluzione da parte di CPLEX in seguito alla sua esecuzione con l'aggiunta del vincolo. Non scegliendo in maniera randomico i lati da selezionare della soluzione precedente, se non dovesse esserci alcun miglioramento del costo e quindi cambiamento della soluzione, i lati selezionati da CPLEX  con il nuovo \textbf{local branching}, sarebbero gli stessi di prima. Per superare tale problema, k viene inizializzata a 2 e, nel caso in cui non dovesse migliorare la soluzione, viene incrementata.\\
Da dati sperimentali, si è appurato come questo metodo aiuti CPLEX a convergere più velocemente alla soluzione ottima e che valori di k superiori a 20 non aiutino a raggiungere risultati migliori.\\
L'aggiunta di un local-branching permette di analizzare in maniera più semplice e veloce lo spazio delle soluzioni. Normalmente per farlo, vengono enumerati gli elementi di questo spazio, generando un numero molto elevatp di possibli soluzioni, considerando che TSP è un problema NP-hard.\\
Definita una soluzione intera ammissibile $x^H$ e utilizzando la distanza di Hamming, vengono definite soluzioni $k-opt$ rispetto ad $x^H$, quelle che hanno distanza k da $x^H$ (vedi Figura \ref{opt}).\\
\begin{figure}[h] 
\begin{center} 
  \includegraphics[scale=0.38]{Images/opt}
  \caption{\footnotesize{Spazio delle soluzioni e distanza di Hamming.}} \label{opt} 
\end{center} 
\end{figure}
Se, invece del local branching, venisse utilizzata l'enumerazione delle soluzioni, si dovrebbero generare per \textbf{k} generico circa $n^k$ soluzioni a distanza k da $x^H$ ed in seguito analizzarle tutte per trovare quella con costo minore e migliore di $x^H$.\\
L'utilizzo del local branching può essere adottato con anche problemi generici e non solo con TSP. Di seguito viene riportato l'aprroccio da adottare per generare tutte le soluzioni a distanza minore o uguale di R dalla soluzione euristica di partenza $x_H$:
\begin{align}
& min \{ c^T x:\;\;Ax=b,\;x\in\{0,1\}^n\} \\ \notag \\
& \underset{j\in E:{x_j}^H=0}\sum{x_j}\;+ \underset{j\in E:{x_j}^H=1}\sum{1-x_j}=\; \leq R
\end{align}
dove \textbf{(3.15)} rappresenta la distanza di Hamming dalla nuova soluzione computata $x$ da $x_H$. L'obiettivo del Soft-fixing è cercare di migliorare il costo della soluzione, guardando quelle più vicine possibili a quella attuale. Nella \textbf{Figura \ref{local_exe}} seguito viene riportato un esmpio di una possibile evoluzione dell'algoritmo nella ricerca dell'ottimo, evindenziandone le soluzioni trovate di volta in volta.
\begin{figure}[h] 
\begin{center} 
  \includegraphics[scale=0.38]{Images/local_exe}
  \caption{\footnotesize{Esempio di esecuzione dell'algoritmo nello spazio delle soluzioni.}} \label{local_exe} 
\end{center} 
\end{figure}
\subsection{Patching algorithm}
Negli algoritmi analizzati nei precedenti capitoli può succedere che CPLEX, prima di restituire la soluzione ottima, computi soluzioni con più componenti connesse. Per evitare che vengano scartate senza essere sfruttate   è possibile utilizzare questo semplice algoritmo euristico che si pone l'obbiettivo di convertirle in una soluzione ammissibile.\\
Date due componenti connesse all'interno della soluzione calcolata, queste vengono unite in una sola grazie all'eliminazione di un ramo ciascuna $\{a, a'\}$ e  $\{b, b'\}$ e alla selezione di altri due rami che fungano da collegamento tra i quattro vertici selezionati ($\{a, b\}$ e $\{a', b'\}$ o $\{a, b'\}$ e $\{a', b\}$) (vedi figura \ref{patching}). 
\begin{figure}[h] 
\begin{center} 
  \includegraphics[scale=0.38]{Images/patching}
  \caption{\footnotesize{Esempio di patching}} \label{patching} 
\end{center} 
\end{figure}
Per scegliere quale ramo per ogni componente connessa sia più conveniente eliminare e con quale sia più conveniente sostituirlo, è necessario minimizzare la variazione di costo che quest'operazione comporterebbe, cioè scegliere il minimo tra: \\\\
$min \begin{cases}
\Delta \;(a,b)\;=c_{ab'} + c_{ba'} - c_{aa'} - c_{bb'}\\ 
\Delta' \;(a,b)\;=c_{ab} + c_{b'a'} - c_{aa'} - c_{bb'}\\ 
\end{cases}
\forall \; a,b \in V : comp(a)\neq comp(b)$
\\\\
L'operazione di fusione di due componenti connesse può essere ripetuta finché la soluzione non diventa ammissibile. In questo modo, al termine, è possibile restituire una soluzione accettabile, ma senza garanzia che sia ottima.\\
Nella nostra implementazione è stato scelto di mantenere invariata ad ogni iterazione una delle componenti connesse e di espanderla fondendola a quella più vicina. Grazie a questa scelta viene minimizzato il numero di rami per cui è necessario modificare la struttura dati che li memorizza.\\
Per poter implementare l'algoritmo è necessario utilizzare due diversi tipi di callback messe a disposizione da CPLEX. La prima appartiene alla tipologia delle lazyconstraintcallback  ed è necessaria per ricevere la soluzione trovata dal programma e rielaborarla. A questa soluzione viene applicato l'algoritmo di patching ed il risultato viene memorizzato all'interno in una struttura dati accessibile anche dalla seconda callback dell'utente. Per garantire che la user-callback sia thread-safe, la struttura dati nominata è stata organizzata di  modo tale che ogni thread accedesse ad una sua specifica porzione.\\
La seconda callback necessaria è un heuristic callback e permette all'utente di suggerire a CPLEX una soluzione da cui proseguire la computazione. Questa soluzione sarà quella memorizzata nella struttura dati dalla prima callback e verrà utilizzata dal
programma solo nel caso in cui sia migliore dell'incumbent attuale.\\
Utilizzando, invece, le generic callback non è necessario implementare due diverse user-callback. è sufficiente invocare la callback in due diversi contesti (uno per ricevere la soluzione di cplex, (CPX\_CALLBACKCONTEXT\_CANDIDATE) e l'altro per suggerire il risultato del patching (CPX\_CALLBACKCONTEXT\_LOCAL\_PROGRESS)). I due diversi casi andranno gestiti dall'interno della funzione stessa.\\
Complessivamente il costo di quest'algoritmo è $O(n^2)$, dove $n$ è il numero di nodi.(verificare)

\begin{algorithm}[H]
\DontPrintSemicolon
\KwIn {$\mathtt{x}$= soluzione di un problema di TSP con più componenti connesse\newline}
\KwOut {$\mathtt{y}$= soluzione intera formata da un'unica componente connessa}
\BlankLine
$\mathtt{n\_comps \gets} numero componenti connesse della soluzione$\;
\BlankLine
$\mathtt{c_1\gets}\{0,...,0\}$\;
\BlankLine
\While{$n\_comps > 1$}{
\BlankLine
$\mathtt{c_1 \gets first\_subtour(}x\mathtt{)}$\;
\BlankLine
$\mathtt{c_2 \gets nearest\_subtour(}c_1\mathtt{)}$\;
\BlankLine
$\mathtt{merge\_component(}c_1,c_2\mathtt{)}$\;
\BlankLine
$\mathtt{update(}n\_comps\mathtt{)}$\;
}
$\mathtt{y \gets} c_1$\;
\caption{Patching}
\end{algorithm}
\chapter{Algoritmi euristici}
In questo capitolo verranno trattati algoritmi euristici che non fanno uso di CPLEX. La necessità di non utilizzare CPLEX si ha per istanze con un numero elevato di nodi (10000-20000 nodi).\\
Per queste istanze la risoluzione del tableau attraverso CPLEX diventerebbe molto un'operazione molto onerosa per via dell'alto numero di variabili che verrano create e su cui verrà svolto il calcolo.\\
Attraverso gli algoritmi euristici, viene computata un'approssimazione della soluzione ottima e spesso però può essere sfruttata inizialmente dal risolutore CPLEX. Ad esempio questo può essere aggiunta prima della computazione, utilizzando la funzione \textit{CPXaddmipstarts()}, o ,se già definita, può essere modificata tramite \textit{CPXchgmipstarts()}.\\
Un algoritmo euristico, affinchè funzioni al meglio, deve essere composto da due fasi che si alternino:
\begin{itemize}
\item{\textbf{Intensificazione o raffinamento}\\
In questa fase la soluzione corrente viene migliorata fino al raggiungimento di un ottimo (locale o globale) nello spazio delle soluzioni.
}
\item{\textbf{Diversificazione}\\
Fase in cui la soluzioni viene perturbata con una politica predefinita affinchè si allontani da un ottimo locale nello spazio delle soluzioni.
}
\end{itemize}


\section{Euristici di costruzione}
Questa prima tipologia di algoritmi euristici è necessaria per la computazione di una prima soluzione ammissibile del problema.
\subsection{Nearest Neighborhood}
Questo algoritmo è basato su un approccio di tipo greedy.
L'algoritmo sceglie un nodo generico tra quelli che compongono il grafo. In seguito, iterativamente seleziona degli archi del grafo secondo il criterio enunciato nella seguente parte.\\
All'iterazione i-esima, vengono analizzati i costi degli archi che hanno un estremo pari al nodo selezionato all'iterazione i-1.\\
Viene selezionato l'arco che tra questi ha costo minore e il nuovo nodo raggiunto viene impostato come punto di partenza per analizzare i costi degli archi all'iterazione successiva (vedi Figura \ref{nearest_neighborhood}). All'ultima iterazione viene scelto l'arco che collega l'ultimo nodo visitato al nodo scelto inizialmente.\\
Il problema di questo algoritmo è che in ogni iterazione viene selezionato esclusivamente il vertice più vicino a quello scelto precedentemente, senza però prevedere la futura evoluzione del ciclo, creato dall'algoritmo.\\
Come in Figura \ref{nearest_neighborhood}, la scelta dell'arco di costo minimo non implica che in seguito venga generata la soluzione ottima. Scegliendo un nodo iniziale differente, viene generato un tour differente.\\
Definito \textbf{n} come il numero di nodi presenti nel grafo, si avranno quindi n soluzioni differenti, ottenute ciascuna attraverso n iterazioni dell'algoritmo. In seguito tra queste possibli soluzioni, verrà selezionata quella di costo minore.\\
\begin{figure}[h] 
\begin{center} 
  \includegraphics[scale=0.4]{Images/nearest_neighborhood}\\ 
  \caption{\footnotesize{Esempio di esecuzione di Nearest Neighborhood.}}
  \label{nearest_neighborhood} 
\end{center} 
\end{figure}
Per migliorare l'algoritmo viene aggiunta una scelta aleatoria, utilizzando il metodo Greedy Adaptive Search Procedure (GASP). Ad ogni iterazione, invece di selezionare l'arco di costo minimo, vengono messi in evidenza i tre archi con costo minore e ne viene selezionato casualmente uno tra questi.\\
Un'implementazione alternativa invece fa uso sia del metodo standard che del metodo GASP. In questo caso la scelta casuale non viene fatta su ogni iterazione ma con una certa periodicità fissata inizialmente.

\subsection{Heuristic Insertion}
L'algoritmo seguente usa un approcco simile al precedente ma prevede la selezione di un ciclo iniziale a cui apportare modifiche, per ottenere una soluzione iniziale ammissibile del problema. Per definire il ciclo di partenza vengono utilizzati diversi metodi. Di seguito sono riportati i due più utilizzati:
\begin{itemize}
\item{\textbf{Selezione di due nodi}\\
Vengono scelti i due nodi più lontani tra loro nel grafo o due nodi casuali e  sono selezionati i due archi orientati che li collegano.
}
\item{\textbf{Inizializzazione geometrica}\\
Nel caso in cui i nodi siano punti 2D, può essere calcolata la convex-hull e utilizzarla come ciclo iniziale. 
}
\end{itemize}
In seguito a tale operazione, questa soluzione viene modificata iterativamente. Per ogni coppia di nodo non appartenente al ciclo \textbf{C}, restituito dall'iterazione precedente, viene calcolato l'extramileage $\Delta_h$ come segue:
$$\Delta_h = \underset{(a,b)\in C}{min} c_{ah}+c_{hb}-c_{ay}$$
con $c_ij$ costo dell'arco che collega i a j (vedi Figura \ref{partial_cycle}).\\
Alla fine di ciascuna iterazione viene aggiunto nel grafo il nodo \textbf{k} che minimizza l'\textbf{extramileage} (vedi Figura \ref{insertion}):\\
$$k = arg\underset{h}{min}\Delta_{h}$$
Un modo per aggiungere aleatorietà a questo algoritmo può essere quello di sfruttare l'approccio GASP nella selezione del minimo extramileage. Il nodo da aggiungere in un'iterazione al grafo, viene selezionato in modo casuale tra i tre con $\Delta$ minore.
\begin{figure}[h] 
\begin{center} 
  \includegraphics[scale=0.2]{Images/partial_cycle}\\ 
  \caption{\footnotesize{Parte del calcolo dell'extramileage del nodo \textbf{h}.}}
  \label{partial_cycle}
\end{center}
\end{figure}
\begin{figure}[h] 
\begin{center} 
  \includegraphics[scale=0.4]{Images/insertion}\\ 
  \caption{\footnotesize{Esempio dell'applicatione di Heuristic insertion.}}
  \label{insertion}
\end{center}
\end{figure}
\section{Algoritmi di raffinamento}
Una volta ottenuto una prima soluzione è necessario migliorarla per avvicinarsi il più possibile all'ottimo. Gli algoritmi utilizzati con questo scopo sono detti \textit{di raffinamento}. Nel capitolo precedente sono già stati descritti due procedimenti di questo tipo, l'Hard Fixing e il Soft Fixing (vedi sottosezioni \ref{hard fixing} e \ref{soft fixing}). In questa sezione verranno invece analizzati algoritmi di raffinamento che non utilizzino funzioni messe a disposizione da CPLEX.
\subsection{Algoritmo di 2 ottimalità}
Nelle soluzioni restituite dagli algoritmi euristici di costruzione sono spesso presenti incroci tra rami che colleghino coppie di nodi appartenenti al circuito. La loro presenza implica sempre la non ottimalità della soluzione, in quanto per le proprietà dei triangoli esisterà sempre una soluzione che eviti l'incrocio e che sia di costo minore (vedi figura \ref{cross}). 
\begin{figure}[h] 
\begin{center} 
  \includegraphics[scale=0.4]{Images/triangle_property}\\ 
  \caption{\footnotesize{Accenno alla dimostrazione della non ottimalità di una soluzione con incrocio.}}
  \label{cross}
\end{center}
\end{figure}
Il procedimento qui descritto si pone l'obiettivo di trovare questa soluzione. Poichè la soluzione che trova si differenzia dalla precedente sempre solo per due rami (quelli che nella prima formavano l'incrocio) viene detto algoritmo di 2 ottimalità.\\
Nell'implementazione del procedimento non è necessario cercare tutti gli incroci della soluzioni per eliminarli, ma è sufficiente analizzare tutte le coppie di rami presenti e verificare se, scambiandole con un'altra coppia ammissibile, si verifica un miglioramento.
\begin{figure}[h] 
\begin{center} 
  \includegraphics[scale=0.5]{Images/switch}\\ 
  \caption{\footnotesize{Esempio di eliminazione di un incrocio.}}
  \label{switch}
\end{center}
\end{figure}
Riferendosi alla figura \ref{switch}, questo viene così calcolato:\\
 $$\Delta = (c_{ac} + c_{bd}) - (c_{ad} + c_{bc})$$
 Solo nel caso in cui risulti $\Delta < 0$ la sostituzione viene memorizzata.\\
In questo modo ogni nuova soluzione appartiene all'intorno di 2 ottimalità, della soluzione precedente, nello spazio delle soluzioni. Aggiornando iterativamente la soluzione si raggiunge un ottimo locale, ovvero in cui non sono possibili miglioramenti nel costo della soluzione, modificando una sola coppia di rami. Questo spostamento è rappresentato in Figura \ref{two_optimality}. 
\begin{figure}[h] 
\begin{center} 
  \includegraphics[scale=0.4]{Images/two_optimality}\\ 
  \caption{\footnotesize{Aggiornamento della soluzione in intorni di 2 ottimalità.}}
  \label{two_optimality}
\end{center}
\end{figure}
Un analogo procedimento viene utilizzato anche nell'algoritmo Soft Fixing, in cui però la dimensione dell'intorno in cui cercare la nuova soluzione può variare (vedi Figura \ref{local_exe}).\\
Poichè per il calcolo del $\Delta$ avviene in tempo costante e deve essere fatto per ogni coppia di rami, il tempo complessivo per la computazione è $O(n^2)$, con $n$ nodi nell'istanza del problema.
\subsection{Algoritmo di 3 ottimalità}
L'algoritmo di 3 ottimalità è analogo a quello analizzato nella sezione precedente, ma considera intorni di 3 ottimalità, nello spazio delle soluzioni, per aggiornare il circuito corrente. In questo caso, quindi, due soluzioni si differenziano per 3 rami, che devono essere scelti con particolare attenzione essendo in maggior numero le possibilità di scelta (vedi Figura \ref{three_optimality}).
\begin{figure}[h] 
\begin{center} 
  \includegraphics[scale=0.4]{Images/three_exchange}\\ 
  \caption{\footnotesize{Due possibile combinazioni di scelta dei nuovi rami da inserire nella soluzioni.}}
  \label{three_optimality}
\end{center}
\end{figure}
Rimanendo comunque in quantità costante, l'algoritmo impiega in tutto $O(n^3)$ (con $n$ numero di nodi) per trovare un ottimo locale, essendo $O(n^3)$ il numero di terne di rami esistenti. Su istante con un alto numero di nodi, questo tempo computazione può essere troppo elevato per il calcolo di una soluzione. 
\section{Meta-euristici}
Gli algoritmi di raffinamento appena visti si occupano di migliorare il più possibile una soluzione già calcolata attraverso meccanismi di local serch. In questo modo, dopo un determinato numero di iterazioni, viene raggiunto un ottimo locale. Le metodologie che vengono descritte in questa sezione cercano di aggiornare questa soluzione allontanandola dall'ottimo locale e convogliandola in un ottimo globale, se possibile. Possono quindi essere descritti come algoritmi euristici per progettare altri euristici, per definizione non sono quindi in grado di garantire il raggiungimento dell'ottimo globale.\\
Questi procedimenti sono più generali di quelli descritti fin'ora, applicabili anche ad istanze di problemi diversi e non solo del commesso viaggiatore. 
\subsection{Multi-starting}
Un primo e intuitivo approccio per allontanarsi da un ottimo locale è quello descritto  dalla politica multi-starting: applicare l'algoritmo di raffinamento scelto a diverse soluzioni iniziali. In questo modo si raggiungono ottimi locali diversi (come mostrato in Figura \ref{multi_starting})e viene restituita la soluzione corrispondente al costo minore. 
\begin{figure}[h] 
\begin{center} 
  \includegraphics[scale=0.5]{Images/multistarting}\\ 
  \caption{\footnotesize{Due possibile esecuzioni di un algoritmo di raffinamento con partenze da due soluzioni diverse.}}
  \label{multi_starting}
\end{center}
\end{figure}
Un aspetto negativo di questo approccio, però, è il fatto che ogni volta che viene selezionata una soluzione di partenza diversa si perdono le informazioni ottenute dalla computazione precedente che ha già raggiunto un ottimo locale.
\subsection{Variable Neighborhood Search \cite{VNS}}
Una volta ottenuta una soluzione corrispondente ad un ottimo locale nella funzione di costo si può cercare di migliorarla analizzando i suoi intorni di ottimalità di raggio crescente. Questo però non garantisce un effettivo cambiamento della soluzione. Il Variable Neighborhood Search (VNS) è un algoritmo che sfrutta questa ricerca. Nel caso non si verifichi un aggiornamento della soluzione prevede che vengano scelti un certi numero di rami randomici da sostituire con altri non appartenenti alla soluzione, in maniera casuale. In questo modo si impone un aggiornamento con una crescita del costo, nella speranza che nel nuovo intorno selezionato sia possibile trovare una soluzione che si allontani dall'iniziale ottimo locale (Figura \ref{VNS}).\\
 \begin{figure}[H] 
\begin{center} 
  \includegraphics[scale=0.5]{Images/VNS_bianco}\\ 
  \caption{\footnotesize{Aggiornamento di una soluzione in un minimo locale.}}
  \label{VNS}
\end{center}
\end{figure}
L'algoritmo termina allo scadere del tempo a disposizione o dopo un determinato numero di iterazione, restituendo la miglior soluzione trovata fino a quel momento.\\
Utilizzando questo approccio gran parte della soluzione di partenza viene conservata evitando di perdere le informazioni elaborate precedentemente all'utilizzo di VNS.
\subsection{Tabu search \cite{Tabu}}%inserire nostra implementazione delle mosse vietate
L'approccio Tabu search fu ideato da Fred W. Glover. Data una soluzione in un ottimo locale dello spazio delle soluzioni, l'idea di Glover permette di aggiornarla anche con una di costo più elevato (generalmente si cerca di minimizzare questo peggioramento). Per evitare che all'iterazione successiva si ritorni nella soluzione di partenza, viene creata una lista di "mosse vietate" , detta Tabu list, che impedisca di raggiungere nuovamente l'ottimo locale. In questo modo la soluzione aumenta di costo per un certo numero di iterazioni, finchè non ricomincia a diminuire per raggiungere un nuovo minimo locale o globale. Nel caso in cui si incontri una soluzione che migliori l'incumbent senza rispettare tutti i vincoli presenti nella Tabu list, l'algoritmo aggiorna ugualmente la soluzione corrente. Questo meccanismo viene detto \textit{Aspiration criterion}.\\ 
Aumentando costantemente di dimensione la lista Tabu si rischia, ad un certo punto, che non sia più possibile aggiornare la soluzione. Per evitare ciò generalmente viene scelta una capienza massima (detta \textit{tenure}) della lista, una volta raggiunta la lista viene aggiornata rispettando la politica FIFO (first in first out). Per far si che l'algoritmo oscilli tra la fase di diversificazione e quella di intensificazione, la tenure viene fatta variare durante le diverse iterazioni tra due valori (massimo e minimo) (vedi Figura \ref{tenure}). Nelle iterazioni in cui è massima si verifica la diversificazione, in quelle in cui è minima l'intensificazione. Se viene applicata questa scelta implementativa l'algoritmo è chiamato \textit{Reactive Tabu Search}.
 \begin{figure}[H] 
\begin{center} 
  \includegraphics[scale=0.35]{Images/tenure}\\ 
  \caption{\footnotesize{Variazione della tenure.}}
  \label{tenure}
\end{center}
\end{figure}
Come per l'algoritmo precedente, il criterio di terminazione è dato dallo scadere del tempo a disposizione o dal raggiungimento del numero massimo di iterazioni scelto, restituendo la miglior soluzione trovata fino a quel momento.\\
\subsection{Simulated annealing}%citazione? trovare articolo?
L'algoritmo simulated annealign, come dice il nome, è ispirato dal processo di temperamento dei metalli, in cui il materiale viene raffreddato molto lentamente e in maniera controllata, affinchè raggiunga la configurazione di minima energia. Analogamente, in questo algoritmo viene scelta una funzione (generalmente esponenziale) che descriva la variazione della "temperatura" $T$. Ad ogni iterazione la soluzione corrente può essere aggiornata con una qualsiasi altra interna all'intorno di 2 ottimalità, di costo minore o maggiore, con una probabilità funzione della variazioni di costo e della temperatura attuale, $f(\Delta costo, T)$. Questo implica che non sia necessario scandire tutto l'intorno, ma che sia sufficiente scegliere in maniera randomica 2 rami da sostituire nella soluzione (Figura \ref{simulated_annealing}).
 \begin{figure}[H] 
\begin{center} 
  \includegraphics[scale=0.6]{Images/simulated_anneling_bianco}\\ 
  \caption{\footnotesize{Esempio di esecuzione dell'algoritmo Simulated Annealing.}}
  \label{simulated_annealing}
\end{center}
\end{figure}
 Con il procedere delle iterazioni, l'aggiornamento ad un costo peggiore avviene sempre meno frequentemente, fino ad ottenere solo aggiornamenti con soluzioni più vantaggiose. Esiste un teorema secondo il quale se la temperatura varia in maniera estremamente lenta ed è consentito effettuare un numero di iterazioni estremamente elevato, questo algoritmo garantisce di trovare l'ottimo globale. Concretamente queste ipotesi sono molto difficili da realizzare, ma è statisticamente comunque possibile dichiarare che l'approccio del simulated annealing restituisce una buona soluzione.
\section{Funzione di costo a scalini}
La funzione obiettivo di un problema può avere, in particolari circostanze, un grafico a scalini, in cui cioè esistano diverse soluzioni dell'istanza con lo stesso costo. In questo caso è consigliabile applicare una politica che utilizzi delle penalità per perturbare la funzione affinché non siano più presenti le zone costanti, che potrebbero causare problemi nell'aggiornamento della soluzione da restituire. Generalmente questa situazione non si verifica con istanze del problema del commesso viaggiatore in cui i costi dipendono dalla distanza euclidea. Nel caso fosse necessario, però, è possibile ammettere che l'algoritmo utilizzato per la risoluzione del problema, restituisca anche soluzioni con più componenti connesse. In questo modo sarebbe possibile scegliere una penalità dipendente dal numero di vincoli violati (ovvero dal numero di subtour presenti) che influenzi il costo della soluzione nel seguente modo:
$$costo\_totale = costo\_archi + M (num\_subtour -1)$$
con M variabile molto grande.
\chapter{Performance}\label{PERF_PROF}
La metrica di confronto, utilizzata nell'analisi degli algoritmi, è il tempo complessivo di creazione e risoluzione del modello. Ciascuna modalità di risoluzione viene applicata a diverse istanze di TSPlib, con un numero differente di nodi.  
\section{Performance variabilty}
Nel corso degli anni '90, gli ingegneri di CPLEX scoprirono che il tempo di risoluzione variasse significativamente in diversi sistemi operativi. Con alcune istanze, le performance migliori si avevano su UNIX mentre con altre su Windows.\\
Il motivo di tale comportamento venne in seguito studiato ed attribuito alla diversa scelta effettuata dai sistemi operativi nel decretare l'ordine delle variabili su cui viene svolto l'albero decisionale.\\
Le scelte svolte inizialmente, nella definizione dei primi nodi dell'albero, si ripercuotono sulla sua successiva evoluzione.\\
Proprio per questo motivo, su alcune istanze, UNIX riusciva a risolvere il problema in tempo minore rispetto a Windows, mentre su altre accadeva l'opposto.\\
Da questi studi, evinse che il Branch and Cut è un sistema caotico e che quindi piccole variazioni delle condizioni iniziali generano grandi differenze nei risultati finali.\\
Per questo motivo, alcuni algoritmi prensentati in questo report, sono stati studiati al variare di alcune condizioni iniziali:
\begin{itemize}
\item{\textbf{Random Seed}\\
definisce il seme da cui CPLEX genera una sequenza di numeri pseudo-casuali (vedi Sezione \ref{param}).\\
Nel momento in cui CPLEX nota che diverse variabili frazionarie hanno lo stesso valore, il risolutore sceglie casualmente su quale di queste applicare il Branch.}
\item{\textbf{Gap}\\
intervallo massimo, tra il valore della migliore funzione obiettivo intera e il valore della funzione di costo del miglior nodo rimanente, che permette di decretare il raggiungimento dell'ottimo secondo CPLEX (vedi Sezione \ref{param}).}
\item{\textbf{Node limit}\\
massimo numero di nodi risolti prima che l'algoritmo termini senza raggiungere l'ottimalità (vedi Sezione \ref{param}).}
\item{\textbf{Populate limit}\\
massimo numero di soluzioni MIP generate per il pool di soluzioni durante ogni chiamata della procedura di popolazione (vedi Sezione \ref{param}).}
\end{itemize}
La variazione del primo di questi parametri permette di apportare significative modifiche al tempo di risoluzione, non modficando la reale ottimalità della soluzione.\\
La variazione degli altri parametri permette invece di ottenere una soluzione meta-euristica, ovvero un'approssimazione più lasca di quella ottima.
 
\section{Analisi tabulare}
Un metodo non molto efficiente per lo studio delle performance degli algoritmi, utilizza una struttura tabulare in cui viene inserita una riga per ogni istanza del problema.\\ Inoltre vengono riportati i tempi di esecuzione degli algoritmi su ognuno dei grafi analizzati. Nell'ultima riga per ciascun algoritmo viene riportata la media geometrica dei suoi tempi di esecuzione (vedi esempio in Tabella \ref{result_table}).\\
Solitamente viene impostato un TIME LIMIT uguale per tutti gli algoritmi. Questo rappresenta nella tabella il valore del tempo di esecuzione per un algoritmo che ha impiegato un ammontare di tempo, maggiore o uguale a TIME LIMIT. Spesso viene dato più peso al TIME LIMIT, inserendolo nella tabella con, ad esempio, peso 10 (ovvero TIME LIMIT*10).\\ La debolezza di tale calcolo delle performance risiede nel fatto che non sempre la media descrive  l'efficienza di un soluzione. Infatti non influisce unicamente il tempo di risoluzione del modello ma anche quello necessario alla sua creazione. 

\begin{table}[h]
\centering
\begin{tabular}{|c|c|c|c|}
\multicolumn{1}{c}{\textbf{Istanza}} & \multicolumn{1}{c}{\textbf{Sequential}} & \multicolumn{1}{c}{\textbf{Flow}} &
\multicolumn{1}{c}{\textbf{Loop}}\\
\hline
\textbf{att48} & {212.3} & {12.5} & {4.3}\\
\hline
{\textbf{...}} & {...} & {...} & {...}\\
\hline
\textbf{a280} & {3200} & {2500.8} & {1300.5}\\
\hline
\hline
\multicolumn{1}{c}{} & \multicolumn{1}{c}{2120.3} & \multicolumn{1}{c}{1800.3}& \multicolumn{1}{c}{1000.4}\\
\end{tabular}
\caption{\footnotesize{Tabella di performance con \textbf{TIME LIMIT=3200}.}}\label{result_table}
\end{table}

\section{Performance profiling}
Questo metodo prevede la classificazione dei tempi di esecuzione degli algoritmi in base al numero percentuale di successi, rispetto a un fattore moltiplicativo (ratio) del tempo di esecuzione (vedi Figura \ref{perf_profile}).\\
L'andamento del performance profile di un algoritmo è monotono crescente. Il valore assunto per ogni ratio dagli algoritmi all'interno del grafico è la percentuale del numero di istanze che l'algoritmo risolve con quel fattore rispetto all'ottimo di quel caso.\\
Spesso questi grafici vengono rappresentati in scala logaritmica per notare al meglio le differenze ed avere una migliore raffigurazione, più semplice da essere analizzata visivamente.
\begin{figure}[h] 
\begin{center} 
  % Requires \usepackage{graphicx} 
  \includegraphics[scale=0.3]{Images/perf_profile} 
  \caption{\footnotesize{Performance profile di due algoritmi.}}
  \label{perf_profile} 
\end{center} 
\end{figure}

Per creare il performance profile degli algoritmi implementati, è stato utilizzato il programma python riportato nella Sezione \ref{perf_profile_py}. 

\section{Analisi degli algoritmi sviluppati}
\subsection{Algoritmi esatti}
\subsubsection{Modelli simmetrici}
\begin{center}
\begin{multicols}{4}
\begin{itemize}
\item{a280.tsp}
\item{ali535.tsp}
\item{att48.tsp}
\item{att532.tsp}
\item{berlin52.tsp}
\item{bier127.tsp}
\item{burma14.tsp}
\item{ch130.tsp}
\item{ch150.tsp}
\item{d198.tsp}
\item{d493.tsp}
\item{d657.tsp}
\item{eil51.tsp}
\item{eil76.tsp}
\item{eil101.tsp}
\item{fl417.tsp}
\item{gil262.tsp}
\item{gr96.tsp}
\item{gr137.tsp}
\item{gr202.tsp}
\item{gr229.tsp}
\item{gr431.tsp}
\item{gr666.tsp}
\item{kroA100.tsp}
\item{kroA150.tsp}
\item{kroA200.tsp}
\item{kroB100.tsp}
\item{kroB150.tsp}
\item{kroB200.tsp}
\item{kroC100.tsp}
\item{kroD100.tsp}
\item{kroE100.tsp}
\item{lin105.tsp}
\item{lin318.tsp}
\item{p654.tsp}
\item{pcb442.tsp}
\item{pr76.tsp}
\item{pr107.tsp}
\item{pr124.tsp}
\item{pr136.tsp}
\item{pr144.tsp}
\item{pr152.tsp}
\item{pr226.tsp}
\item{pr264.tsp}
\item{pr299.tsp}
\item{pr439.tsp}
\item{rat99.tsp}
\item{rat195.tsp}
\item{rat575.tsp}
\item{rat783.tsp}
\item{rd100.tsp}
\item{rd400.tsp}
\item{st70.tsp}
\end{itemize}
\end{multicols}
\end{center}

\begin{figure}[h] 
\begin{center} 
  % Requires \usepackage{graphicx} 
  \includegraphics[scale=0.8]{Images/pp_exact}\\ 
  \caption{\footnotesize{Performance profile degli algoritmi esatti.}}
  \label{perf_profile} 
\end{center} 
\end{figure}

\subsubsection{Modelli compatti}
\begin{center}
\begin{multicols}{3}
\begin{itemize}
\item{att48.tsp}
\item{berlin52.tsp}
\item{burma14.tsp}
\item{eil101.tsp}
\item{eil51.tsp}
\item{eil76.tsp}
\item{gr96.tsp}
\item{kroA100.tsp}
\item{kroB100.tsp}
\item{kroB150.tsp}
\item{kroC100.tsp}
\item{kroD100.tsp}
\item{kroE100.tsp}
\item{pr124.tsp}
\item{pr136.tsp}
\item{pr76.tsp}
\item{rat99.tsp}
\item{rd100.tsp}
\item{st70.tsp}
\item{ulysses16.tsp}
\end{itemize}
\end{multicols}
\end{center}

\begin{figure}[h] 
\begin{center} 
  % Requires \usepackage{graphicx} 
  \includegraphics[scale=0.8]{Images/pp_compact}\\ 
  \caption{\footnotesize{Performance profile degli algoritmi compatti.}}
  \label{perf_profile} 
\end{center} 
\end{figure}

\subsection{Algoritmi math-euristici}

\subsection{Algoritmi euristici}
\subsubsection{Multi-start}\label{construction_perf}
\begin{center}
\begin{multicols}{4}
\begin{itemize}
\item{a280.tsp}
\item{ali535.tsp}
\item{att532.tsp}
\item{d1291.tsp}
\item{d1665.tsp}
\item{d2103.tsp}
\item{d493.tsp}
\item{d657.tsp}
\item{dsj1000.tsp}
\item{fl1400.tsp}
\item{fl1577.tsp}
\item{fl417.tsp}
\item{gil262.tsp}
\item{gr431.tsp}
\item{gr666.tsp}
\item{lin318.tsp}
\item{nrw1379.tsp}
\item{p654.tsp}
\item{pcb1173.tsp}
\item{pcb442.tsp}
\item{pr1002.tsp}
\item{pr299.tsp}
\item{pr439.tsp}
\item{rat575.tsp}
\item{rat783.tsp}
\item{rd400.tsp}
\item{rl1304.tsp}
\item{rl1323.tsp}
\item{rl1889.tsp}
\item{u1060.tsp}
\item{u1432.tsp}
\item{u1817.tsp}
\item{u574.tsp}
\item{u724.tsp}
\item{vm1084.tsp}
\item{vm1748.tsp}
\end{itemize}
\end{multicols}
\end{center}

\begin{figure}[h] 
\begin{center} 
  % Requires \usepackage{graphicx} 
  \includegraphics[scale=0.8]{Images/pp_construction}\\ 
  \caption{\footnotesize{Confronto dei vari multistart in base all'algoritmo di costruzione.}}
  \label{perf_profile} 
\end{center} 
\end{figure}

\subsubsection{Algoritmi meta-euristici}
\begin{center}
\begin{multicols}{3}
\begin{itemize}
\item{d1291.tsp}
\item{d1655.tsp} 
\item{d2103.tsp} 
\item{dsj1000.tsp}
\item{fl1400.tsp} 
\item{fl1577.tsp} 
\item{nrw1379.tsp} 
\item{pcb1173.tsp} 
\item{pr1002.tsp}
\item{pr2392.tsp}
\item{rl1304.tsp}
\item{rl1323.tsp}
\item{rl1889.tsp}
\item{u1060.tsp} 
\item{u1432.tsp}
\item{u1817.tsp}
\item{u2152.tsp}
\item{u2319.tsp}
\item{vm1084.tsp}
\item{vm1748.tsp}
\end{itemize}
\end{multicols}
\end{center}

\begin{figure}[h] 
\begin{center} 
  % Requires \usepackage{graphicx} 
  \includegraphics[scale=0.8]{Images/pp_heuristic}\\ 
  \caption{\footnotesize{Confronto dei costi ottenuti algoritmi meta-euristici sviluppati utilizzando il Nearest Neighborhood come algoritmo di costruzione.}}
  \label{perf_profile} 
\end{center} 
\end{figure}

\begin{figure}[h] 
\begin{center} 
  % Requires \usepackage{graphicx} 
  \includegraphics[scale=0.8]{Images/cost_vns}\\ 
  \caption{\footnotesize{Confronto dei vari multistart in base all'algoritmo di costruzione.}}
  \label{perf_profile} 
\end{center} 
\end{figure}

\begin{figure}[h] 
\begin{center} 
  % Requires \usepackage{graphicx} 
  \includegraphics[scale=0.8]{Images/cost_tabu}\\ 
  \caption{\footnotesize{Confronto dei vari multistart in base all'algoritmo di costruzione.}}
  \label{perf_profile} 
\end{center} 
\end{figure}

\begin{figure}[h] 
\begin{center} 
  % Requires \usepackage{graphicx} 
  \includegraphics[scale=0.8]{Images/cost_sa}\\ 
  \caption{\footnotesize{Confronto dei vari multistart in base all'algoritmo di costruzione.}}
  \label{perf_profile} 
\end{center} 
\end{figure}

\begin{figure}[h] 
\begin{center} 
  % Requires \usepackage{graphicx} 
  \includegraphics[scale=0.8]{Images/cost_genetic}\\ 
  \caption{\footnotesize{Confronto dei vari multistart in base all'algoritmo di costruzione.}}
  \label{perf_profile} 
\end{center} 
\end{figure}
\appendix
\chapter{TSPlib}
Un'istanza di tale problema viene definita normalmente da un grafo, per cui ad ogni nodo viene associata un numero intero (Ex. $\Pi = \{1,2,3,..,n\}$). \\
Una soluzione del problema è una sequenza di nodi che corrisponde ad una permutazione dell'istanza (es. $S = \{x_1,x_2,...,x_n\}$ tale che $x_i=x_j\;\;x_i \in \Pi\;\forall\;x_i\in S\;\wedge\; x_i!=x_j\;\forall\;i\neq j$). Poichè in questa variante non esiste alcuna origine, ogni tour può essere descritto da due versi di percorrenza e l'origine può essere un nodo qualsiasi del grafo.\\
La rappresentazione di tali istanze è stata svolta attraverso l'utilizzo del programma Gnuplot. Per avere dettagli riguardanti il suo utilizzo vedi Sezione \ref{gnuplot}.

Le istanze del problema, analizzate durante il corso, sono punti dello spazio 2D, identificati quindi da due coordinate ($x$,$y$).
Per generare istanza enormi del problema, si utilizza un approccio particolare in cui viene definito un insieme di punti a partire da un'immagine già esistente.\\
La vicinanza dei punti generati dipende dalla scala di grigi all'interno dell'immagine (es. generazione di punti a partire dal dipinto della Gioconda\cite{monnalisa}).\\
Le istanze che vengono elaborate dai programmi, creati durante il corso, utilizzano il template \textbf{TSPlib}. Di seguito viene riportato il contenuto di un file di questa tipologia.
 
\lstinputlisting[caption={\footnotesize{esempio.tsp}}, style=code, firstnumber=1, firstline=1, lastline=12, label=tsp_instance]{Source/esempio.tsp}

Le parole chiave più importanti, contenute in questi file \ref{tsp_instance}, sono:
\begin{itemize}
\item{\textbf{NAME}\\
seguito dal nome dell'istanza TSPlib}
\item{\textbf{COMMENT}\\
seguito da un commento associato all'istanza}
\item{\textbf{TYPE}\\
seguito dalla tipologia dell'istanza}
\item{\textbf{DIMENSION}\\
seguito dal numero di nodi nel grafo ($num\_nodi$)}
\item{\textbf{EDGE\_WEIGHT\_TYPE}\\
seguito dalla specifica del tipo di calcolo che viene effettuato per ricavare il costo del tour}
\item{\textbf{NODE\_COORD\_SECTION}\\
inizio della sezione composta di $num\_nodi$ righe in cui vengono riportate le caratteriste di ciascun nodo, nella forma seguente:
\begin{lstlisting}[linewidth=250pt,basicstyle=\footnotesize\sffamily,]     
indice_nodo  coordinata_x  coordinata_y
\end{lstlisting}}
\item{\textbf{EOF}\\
decreta la fine del file}
\end{itemize}
\chapter{ILOG CPLEX}
In questa sezione verranno approfondite alcune funzioni di CPLEX necessarie ad implementare gli algoritmi descritti nei capitoli precedenti.

\section{Funzionamento}
Per poter utilizzare gli algoritmi di risoluzione forniti da CPLEX è necessario costruire il modello matematico del problema, legato all'istanza precedentemente descritta.\\
CPLEX ha due meccanismi di acquisizione dell'istanza:
\begin{enumerate}
\item{\textbf{modalità interattiva:}\\
in cui il modello viene letto da un file precedentemente generato (\textit{model.lp})}
\item{\textbf{creazione nel programma:}\\
il modello viene creato attraverso le API del linguaggio usato per la scrittura del programma}
\end{enumerate}

Le strutture utilizzate da CPLEX sono due (vedi Figura \ref{strutture_cplex}):
\begin{itemize}
\item{\textbf{ENV:} contiene i parametri necessari all'esecuzione e al salvataggio dei risultati}
\item{\textbf{LP:} contiene il modello che viene analizzato da CPLEX durante la computazione del problema di ottimizzazione}
\end{itemize}

\begin{figure}[h] 
\begin{center} 
  \includegraphics[scale=0.7]{Images/cplex_structs}\\ 
  \caption{\footnotesize{Strutture CPLEX}}
  \label{strutture_cplex} 
\end{center} 
\end{figure}

Ad ogni ENV è possibile associare più LP, in modo da poter risolvere in parallelo più problemi di ottimizzazione, ma nel nostro caso ne sarà sufficiente solo uno.\\
Per convenzione è stato deciso di etichettare i rami $(i,j)$ dell'istanza rispettando la proprietà $i<j$. In Figura \ref{Indici_matrice} è riportato lo schema degli indici che vengono utilizzati per etichettare le variabili.\\
In questa figura le celle $(i,j)$ bianche, sono quelle effettivamente utilizzate per indicare un arco secondo la convenzione. Il numero all'interno di queste caselle rappresenta invece l'ordine in cui queste variabili vengono inserite nel modello e quindi gli indici associati da CPLEX per accedere alla soluzione.\\
\begin{figure}[h] 
\begin{center} 
  \includegraphics[scale=0.6]{Images/indices_matrix}\\ 
  \caption{\footnotesize{Indici della matrice}}
  \label{Indici_matrice} 
\end{center} 
\end{figure}

\section{Funzioni}
\subsection{Costruzione e modifica del modello}
Per poter costruire il modello da analizzare, come prima cosa, è necessario creare un puntatore alle due strutture dati utilizzate da CPLEX.
\lstinputlisting[caption={\footnotesize{modelTSP.txt}}, style=code, firstnumber=1, firstline=27, lastline=29, label=tsp_model, language=c]{Source/modelTSP.txt}
La funzione alla riga 2 alloca la memoria necessaria e riempie la struttura con valori di default. Nel caso in cui non termini con successo memorizza un codice d'errore in \textit{error}.\\
La funziona invocata nella riga successiva, invece, associa la struttura LP all'ENV che gli viene fornito. Il terzo parametro passato, nell'esempio "TSP", sarà il nome del modello. Al termine di queste operazioni verrà quindi creato un modello vuoto. All'interno del nostro programma per inizializzarlo è stata costruita la seguente funzione:
\begin{center}
\begin{tabular}{c}
\begin{lstlisting}[linewidth=370pt, basicstyle=\footnotesize\sffamily,] 
void cplex_build_model(tsp_instance* tsp_in, CPXENVptr env, CPXLPptr lp);
\end{lstlisting}
\end{tabular}
\end{center}
\begin{table}[h]
\centering
\begin{tabular}{rl}
\textbf{tsp\_in: } & {puntatore alla struttura che contiene l'istanza del problema (letta dal file TSPlib)} \\
\textbf{env: } & {puntatore alla struttura ENV precedentemente creata}\\
\textbf{lp: } & {puntatore alla struttura LP  precedentemente creata}\\
\end{tabular}
\end{table}
All'interno di \textbf{cplex\_build\_model()} viene aggiunta una colonna alla volta al modello, definendo quindi anche la funzione obiettivo. Le variabili aggiunte corrispondono agli archi del grafo e per ciascuno di questi viene calcolato il costo come distanza euclidea. La funzione necessaria ad inserire colonne e definire la funzione di costo è la seguente:
\begin{center}
\begin{tabular}{c}
\begin{lstlisting}[linewidth=400pt, basicstyle=\footnotesize\sffamily,] 
int CPXnewcols( CPXCENVptr env, CPXLPptr lp, int ccnt, double const * obj, 
		double const * lb, double const * ub, char const * xctype, char ** colname);    
\end{lstlisting}
\end{tabular}
\end{center}
\begin{table}[h]
\begin{tabular}{rl}
\textbf{env:} & {puntatore alla struttura ENV precedentemente creata}\\
\textbf{lp:} & {puntatore alla struttura LP precedentemente creata}\\
\textbf{ccnt:} & {numero di colonne da inserire} \\    
\textbf{obj:} & {vettore dei costi relativi agli archi da inserire} \\
\textbf{lb:} & {vettore contenente i lower bound dei valori assumibili dalle variabili da}\\
&{inserire}\\            
\textbf{ub:} & {vettore contenente gli upper bound dei valori assumibili dalle variabili da}\\
&{inserire}\\
\textbf{xctype:} & {vettore contenente la tipologia delle variabili da inserire}\\
\textbf{colname:} & {vettore di stringhe contenenti i nomi delle variabili da inserire}\\
\textbf{Return Value:} & {0 in caso di successo, un valore diverso da 0 se si verifica un errore}\\
\end{tabular}
\end{table}
La generica colonna \textbf{i}, aggiunta dalla funzione, sarà definita dalle informazioni contenute all'interno della posizione \textbf{i} degli array, ricevuti come parametri. Nel programma elaborato durante il corso, viene aggiunta una colonna alla volta all'interno del modello. Per far ciò, è necessario comunque utilizzare riferimenti alle informazioni da inserire, in modo da ovviare il problema riguardante la tipologia di argomenti richiesti, che sono array. Ad esempio, nel nostro caso, la tipologia di una nuova variabile inserita sarà un riferimento al carattere \textbf{'B'}, che la identifica come binaria.\\
Per poter inserire il primo insieme di vincoli del problema\\
$$
\underset{e\in \delta(v)}\sum{\;x_e} = 2\;\;\;\;\;\;\;\;\;\;\;\;\;\;\;\;\;\;\forall\;v\in V \\\\
$$
\\
viene invece sfruttata la seguente funzione:
\begin{center}
\begin{tabular}{c}
\begin{lstlisting}[linewidth=380pt, basicstyle=\footnotesize\sffamily,]  
int CPXnewrows( CPXCENVptr env, CPXLPptr lp, int rcnt, double const * rhs,
		char const * sense, double const * rngval, char ** rowname);   
\end{lstlisting}
\end{tabular}
\end{center}
\begin{table}[h]
\centering
\begin{tabular}{rl}
\textbf{env:} & {puntatore alla struttura ENV precedentemente creata}\\
\textbf{lp:} & {puntatore alla struttura LP precedentemente creata}\\
\textbf{rcnt:} & {numero di righe (vincoli) da inserire}\\
\textbf{rhs:} & {vettore dei termini noti dei vincoli}\\
\textbf{sense:} & {vettore di caratteri che specifica il tipo di vincoli da inserire.}\\
&{Ogni carattere può assumere:}\\
&{\textit{'L'} per vincolo $\leq$}\\
&{\textit{'E'} per vincolo $=$}\\
&{\textit{'G'} per vincolo $\geq$}\\
&{\textit{'R'} per vincolo definito in un intervallo}\\
\textbf{rngval:} & {vettore di range per i valori di ogni vincolo (nel nostro caso è NULL)}\\
\textbf{rowname} & {vettore di stringhe contenenti i nomi delle variabili da inserire}\\
\textbf{Return Value:} & {0 in caso di successo, un valore diverso da 0 se si verifica un errore}\\
\end{tabular}
\end{table}
In modo analogo all'inserimento delle colonne, nel nostro programma viene aggiunta una riga alla volta nel modello. L'\textbf{i}-esima riga aggiunta corrisponderà al vincolo imposto sul nodo \textbf{i}-esimo, imponendo a 1 il coefficiente della variabile $x_{k,j}$ se $k=i \;\wedge j=i$ per ogni variabile del modello. In questo modo però viene aggiunto un vincolo in cui è necessario cambiare i coefficienti delle variabili che ne prendono parte. Per fare ciò è necessaria la funzione:
\begin{center}
\begin{tabular}{c}
\begin{lstlisting}[linewidth=380pt, basicstyle=\footnotesize\sffamily,]   
int CPXchgcoef(CPXCENVptr env, CPXLPptr lp, int i, int j, double newvalue);
\end{lstlisting}
\end{tabular}
\end{center}
\begin{table}[h]
\centering
\begin{tabular}{rl}
\textbf{env:} & {puntatore alla struttura ENV precedentemente creata}\\
\textbf{lp:} & {puntatore alla struttura LP precedentemente creata}\\
\textbf{i:} & {intero che specifica l'indice della riga in cui modificare il coefficiente}\\
\textbf{j:} & {intero che specifica la colonna in cui si trova la variabile di cui modificare}\\
&{il coefficiente}\\
\textbf{newvalue:} & {nuovo valore del coefficiente}\\
\textbf{Return Value:} & {0 in caso di successo, un valore diverso da 0 se si verifica un errore}\\
\end{tabular}
\end{table}
L'utilizzo di questa metodo per inserire nuovi vincoli è però considerato inefficiente. Al suo posto è consigliato l'utilizzo di una funzione che inserisca il vincolo con già i coefficienti delle variabili impostati al valore corretto:
\begin{center}
\begin{tabular}{c}
\begin{lstlisting}[linewidth=400pt, basicstyle=\footnotesize\sffamily,]  
int CPXaddrows( CPXCENVptr env, CPXLPptr lp, int ccnt, int rcnt, int nzcnt,
		double const * rhs, char const * sense, int const * rmatbeg, 
		int const * rmatind, double const * rmatval, char ** colname, 
		char ** rowname );   
\end{lstlisting}
\end{tabular}
\end{center}
\begin{table}[h]
\centering
\begin{tabular}{rl}
\textbf{env:} & {puntatore alla struttura ENV precedentemente creata}\\
\textbf{lp:} & {puntatore alla struttura LP precedentemente creata}\\
\textbf{ccnt:} & {numero di nuove colonne che devono essere aggiunte}\\
\textbf{rcnt:} & {numero di nuove righe che devono essere aggiunte}\\
\textbf{nzcnt:} & {numero di coefficienti non nulli nel vincolo aggiunto}\\
\textbf{rhs:} & {vettore con i termini noti per ogni vincolo da aggiungere}\\
\textbf{sense:} & {vettore con il tipo di vincoli da aggiungere, scelto tra:}\\
&{\textit{'L'} per vincolo $\leq$}\\
&{\textit{'E'} per vincolo $=$}\\
&{\textit{'G'} per vincolo $\geq$}\\
&{\textit{'R'} per vincolo definito in un intervallo}\\
\textbf{rmatbeg:} & {vettore per definire le righe da aggiungere}\\
\textbf{rmatind:} & {vettore per definire le righe da aggiungere}\\
\textbf{rmatval:} & {vettore per definire le righe da aggiungere}\\
\textbf{colname:} & {vettore contenente i nomi delle nuove colonne}\\
\textbf{rowname:} & {vettore contenente i nomi dei nuovi vincoli}\\
\textbf{Return Value:} & {0 in caso di successo, un valore diverso da 0 se si verifica un errore}\\
\end{tabular}
\end{table}
Per rimuovere invece delle righe, viene utilizzata la seguente funzione:
\begin{center}
\begin{tabular}{c}
\begin{lstlisting}[linewidth=330pt, basicstyle=\footnotesize\sffamily,]     
int CPXdelrows( CPXCENVptr env, CPXLPptr lp, int begin, int end );
\end{lstlisting}
\end{tabular}
\end{center}
\begin{table}[h]
\centering
\begin{tabular}{rl}
\textbf{env:} & {puntatore alla struttura ENV precedentemente creata}\\
\textbf{lp:} & {puntatore alla struttura LP precedentemente creata}\\
\textbf{begin:} & {indice numerico della prima riga da cancellare}\\
\textbf{end:} & {indice numerico dell'ultima riga da cancellare}\\
\textbf{Return Value:} & {0 in caso di successo, un valore diverso da 0 se si verifica un errore}\\
\end{tabular}
\end{table} 
\vspace{2 cm}
Per poter impostare una variabile $x_{i,j}$ ad una valore fissato è necessario rendere i suoi lower e upper bound alla quantità voluta. Per cambiare questi parametri viene utilizzata la seguente funzione: 
\begin{center}
\begin{tabular}{c}
\begin{lstlisting}[linewidth=375pt, basicstyle=\footnotesize\sffamily,]   
int CPXchgbds(CPXCENVptr env, CPXLPptr lp, int cnt, const int * indices, 
		const char * lu, const double * bd); 
\end{lstlisting}
\end{tabular}
\end{center}
\begin{table}[h]
\centering
\begin{tabular}{rl}
\textbf{env:} & {puntatore alla struttura ENV}\\
\textbf{lp:} & {puntatore alla struttura LP}\\
\textbf{cnt:} & {numero totale di bound da cambiare}\\
\textbf{indices:} & {vettore con gli indice delle colonne corrispondenti alle variabili}\\
&{di cui cambiare il bound}\\
\textbf{lu:} & {array di caratteri che specificano il bound da modificare,}\\
&{a scelta tra:}\\
&{\textit{'U'} per upper bound}\\
&{\textit{'L'} per lower bound}\\
&{\textit{'B'} per entrambi}\\
\textbf{bd:} & {vettore con i nuovi valori}\\
\textbf{Return Value:} & {0 in caso di successo, un valore diverso da 0 se si verifica un errore}\\
\end{tabular}
\end{table}
\subsection{Calcolo della soluzione}
Per ottenere la soluzione ottima del problema di ottimizzazione correlato al modello definito in CPLEX, vengono utilizzate due fasi:
\begin{itemize}
\item{\textbf{Risoluzione del problema di ottimizzazione}\\\\
\begin{tabular}{c}
\begin{lstlisting}[linewidth=220pt, basicstyle=\footnotesize\sffamily,]
int CPXmipopt(CPXCENVptr env, CPXLPptr lp);
\end{lstlisting}
\end{tabular}
\begin{table}[h]
\centering
\begin{tabular}{rl}
\textbf{env:} & {puntatore alla struttura ENV precedentemente creata}\\
\textbf{lp:} & {puntatore alla struttura LP precedentemente creata}\\
\textbf{Return Value:} & {0 in caso di successo, un valore diverso da 0 se si verifica un errore}\\
\end{tabular}
\end{table}
}
\item{\textbf{Ottenimento della soluzione}\\\\
\begin{tabular}{c}
\begin{lstlisting}[linewidth=380pt, basicstyle=\footnotesize\sffamily,]
int CPXgetmipx (CPXENVptr env, CPXLPptr lp, double *x, int begin, int end);
\end{lstlisting}
\end{tabular}
\vspace{2 cm}
\begin{table}[h]
\centering
\begin{tabular}{rl}
\textbf{env:} & {puntatore alla struttura ENV precedentemente creata}\\
\textbf{lp:} & {puntatore alla struttura LP precedentemente creata}\\
\textbf{x:} & {puntatore a un vettore di double in cui verranno salvati i valori}\\
& {delle variabili ottenuti dalla soluzione ottima}\\
\textbf{begin:} & {primo indice della variabile di cui si vuole memorizzare ed analizzare}\\
&{ il valore}\\
\textbf{end:} & {indice dell'ultima variabile di cui si vuole memorizzare ed analizzare}\\
&{il valore}\\
\textbf{Return Value:} & {0 in caso di successo, un valore diverso da 0 se si verifica un errore}\\
\end{tabular}
\end{table}
\\Questa funzione salva in $x$ tutte le variabili che hanno indice $i\in [begin, end]$ e quindi $x$ deve essere un vettore di almeno $end-begin+1$ valori. Nel nostro programma, vengono analizzati i valori di tutte le variabili definite.\\
Per questo motivo \textbf{begin = 0} e 
\textbf{end = num\_colonne - 1}\footnote{numero di variabili=CPXgetnumcols(env,lp);}\footnote{numero di vincoli=CPXgetnumrows(env,lp);}.\\
In seguito il nostro programma analizza la correttezza della soluzione svolgendo la verifica su:
\begin{itemize}
\item{\textit{valori assunti dalle variabili}\\
ciascun $x_{i,j}$ assume valore $0$ o $1$ con una tolleranza di $\epsilon=10^{-5}$}
\item{\textit{grado di ciascun nodo}\\
il tour è composto al massimo da due archi che tocchino lo stesso nodo}
\end{itemize}
}
\item{\textbf{Gap relativo}\\
La seguente funzione permette di ottenere il gap relativo della funzione obiettivo per un'ottimizzazione MIP.\\
\begin{center}
\begin{tabular}{c}
\begin{lstlisting}[linewidth=350pt, basicstyle=\footnotesize\sffamily,]
int CPXgetmiprelgap( CPXCENVptr env, CPXCLPptr lp, double * gap_p );
\end{lstlisting}
\end{tabular}
\end{center}
\begin{table}[h]
\centering
\begin{tabular}{rl}
\textbf{env:} & {puntatore alla struttura ENV precedentemente creata}\\
\textbf{lp:} & {puntatore alla struttura LP precedentemente creata}\\
\textbf{gap\_p:} & {puntatore in cui verrà salvato il gap}\\
\textbf{Return Value:} & {0 in caso di successo, un valore diverso da 0 se si verifica un errore}\\
\end{tabular}
\end{table}
Per un problema di minimizzazione il gap relativo viene calcolato come:
$$\frac{bestinteger - bestobjective}{10^{-10}+|bestinteger|}$$
dove \textbf{bestinteger} è il valore restituito dalla funzione \textbf{CPXgetobjval()} mentre               \textbf{bestobjective} da \textbf{CPXgetbestobjval()}.
}
\end{itemize}

\subsection{Lazy constraints}
Nel caso in cui si voglia verificare il soddisfacimento di un vincolo solo al termine della computazione della soluzione, è necessario inserire un \textbf{"lazy constraint"}. Questi vincoli vengono 
dichiarati in fase di costruzione del modello e aggiunti ad un pull. Per fare ciò viene utilizzata la seguente funzione:
\begin{center}
\begin{tabular}{c}
\begin{lstlisting}[linewidth=390pt, basicstyle=\footnotesize\sffamily,]  
int CPXaddlazyconstraints( CPXCENVptr env, CPXLPptr lp, int rcnt, int nzcnt, 
		double const * rhs, char const * sense, int const * rmatbeg, 
		int const * rmatind, double const * rmatval, char ** rowname );
\end{lstlisting}
\end{tabular}
\end{center}
\begin{table}[h]
\centering
\begin{tabular}{rl}
\textbf{env:} & {puntatore alla struttura ENV precedentemente creata}\\
\textbf{lp:} & {puntatore alla struttura LP precedentemente creata}\\
\textbf{rcnt:} & {numero di vincoli da inserire}\\
\textbf{nzcnt:} & {numero di coefficienti non nulli nel vincolo}\\ 
\textbf{rhs:} & {vettore dei termini noti dei vincoli}\\
\textbf{sense:} & {vettore di caratteri che specifica il tipo di vincoli da inserire.}\\
&{Ogni carattere può assumere:}\\
&{\textit{'L'} per vincolo $\leq$}\\
&{\textit{'E'} per vincolo $=$}\\
&{\textit{'G'} per vincolo $\geq$}\\
&{\textit{'R'} per vincolo definito in un intervallo}\\
\textbf{rmatbeg:} & {vettore con le posizione iniziali dei coefficienti nei vincoli}\\
\textbf{rmatind:} & {vettore di vettori contenenti gli indici delle variabili appartenenti al vincolo}\\
\textbf{rmatval:} & {vettore di vettori con i coefficienti delle variabili del vincolo}\\
\textbf{rowname:} & {vettore con i nomi dei vincoli}\\
\textbf{Return Value:} & {0 in caso di successo, un valore diverso da 0 se si verifica un errore}\\
\end{tabular}
\end{table}
In modo analogo alle due funzioni precedentemente descritte per l'aggiunta di righe e colonne, nel nostro modello viene inserito un vincolo per volta. Per impostare correttamente i coefficienti delle variabili presenti nel vincolo, vengono sfruttati i due array \textit{rmatinds} e \textit{rmatval}. Come rappresentato in Figura \ref{lazy_constraints}, all'interno della posizione \textit{i}-esima del vettore di indici è presente la posizione dell'\textit{i}-esima variabile del vincolo da inserire (nell'esempio in figura $rmatinds[i]=j$). Mentre l'\textit{i}-esima posizione del vettore di valori contiene il corrispondente  coefficiente (in questo caso $c_j$).
\begin{figure}[h] 
\begin{center} 
  % Requires \usepackage{graphicx} 
  \includegraphics[scale=0.5]{Images/lazy_constraints} 
  \caption{\footnotesize{Array lazy constraints}}
  \label{lazy_constraints} 
\end{center} 
\end{figure}
\subsection{Lazy Constraint Callback}
Per poter utilizzare una lazy constraint callback, precedentemente implementata, all'interno del programma, prima di tutto è necessario installarla. Questo viene fatto attraverso la seguente funzione:
\begin{center}
\begin{tabular}{c}
\begin{lstlisting}[linewidth=385pt, basicstyle=\footnotesize\sffamily,] 
int CPXsetlazyconstraintcallbackfunc( CPXENVptr env,
		int(CPXPUBLIC *lazyconcallback)(CALLBACK_CUT_ARGS), void * cbhandle);    
\end{lstlisting}
\end{tabular}
\end{center}
\begin{table}[h]
\centering
\begin{tabular}{rl}
\textbf{env:} & {puntatore alla struttura ENV}\\
\textbf{lazyconcallback:} & {puntatore alla callback chiamata}\\
\textbf{cbhandle:} & {puntatore ad una struttura dati contenente le informazioni}\\
& {da passare alla callback}\\
\textbf{Return Value:} & {0 in caso di successo, un valore diverso da 0 se si verifica un errore}\\
\end{tabular}
\end{table}
Una volta installata la callback è necessario cambiare l'impostazione del numero di thread utilizzati dal programma. Infatti CPLEX, non sapendo se la funzione implementata dall'utente è thread safe, impedisce lo svolgimento di elaborazioni in parallelo con le callback. A meno che questo non venga esplicitamente dichiarato dall'utente con l'impostazione del corrispondente parametro.
Per questo può tornare utile la seguente funzione, che restituisce il numero di core presenti nel computer:\\
\begin{center}
\begin{tabular}{c}
\begin{lstlisting}[linewidth=280pt, basicstyle=\footnotesize\sffamily,]
int CPXgetnumcores(CPXCENVptr env, int * numcores_p); 
\end{lstlisting}
\end{tabular}
\end{center}
\begin{table}[h]
\centering
\begin{tabular}{rl}
\textbf{env:} & {puntatore ad una struttura ENV}\\
\textbf{numcores\_p:} & {puntatore alla variabile in cui scrivere il numero di core}\\ 
\textbf{Return Value:} & {0 in caso di successo, un valore diverso da 0 se si verifica un errore}\\           
\end{tabular}
\end{table} 
Come descritto nella sezione dedicata, le callback sono funzioni lasciate appositamente vuote da CPLEX, affinché l'utente possa implementarle in maniera personalizzata. Hanno però una dichiarazione standard, qui riportata: 
\begin{center}
\begin{lstlisting}[linewidth=400pt, basicstyle=\footnotesize\sffamily,]     
static int CPXPUBLIC name_function(CPXCENVptr env, void* cbdata, int wherefrom,
		 void* cbhandle, int* useraction_p);
\end{lstlisting}
\end{center}
\begin{table}[h]
\centering
\begin{tabular}{rl}
\textbf{env:} & {puntatore una struttura ENV}\\
\textbf{cbdata:} & {puntatore che contiene specifiche informazioni per la callback}\\
\textbf{wherefrom:} & {contiene dove è stata invocata la callback durante l'ottimizzazione} \\ 
\textbf{cbhandle:} & {puntatore a dati privati dell'utente} \\
\textbf{useraction\_p:} & {specifica le azioni da eseguire al termine della callback:}\\
& {CPX\_CALLBACK\_DEFAULT: usa il nodo di CPLEX selezionato}\\
& {CPX\_CALLBACK\_FAIL: esci dell'ottimizzazione}\\
& {CPX\_CALLBACK\_SET: usa il nodo selezionato come definito}\\  
& {nel valore di ritorno}\\ 
\textbf{Return Value:} & {0 in caso di successo, un valore diverso da 0 se si verifica un errore}\\                      
\end{tabular}
\end{table} 
Nell'implementarla bisogna fare particolare attenzione a renderla thread safe, se si vuole utilizzarla su più processi in parallelo. Infatti, nel caso in cui il programma lavorasse contemporaneamente con più processori, non si devono verificare interferenze di accesso agli stessi dati da parte di invocazioni diverse della callback. Quest'aspetto è lasciato a completa gestione dell'utente.\\
%CONTROLLARE COSA VUOL DIRE CPXPUBLIC
Per avere accesso alle variabili utilizzate dal nodo che invoca la callback è possibile chiamare la seguente funzione:
\begin{center}
\begin{tabular}{c}
\begin{lstlisting}[linewidth=410pt, basicstyle=\footnotesize\sffamily,]     
int CPXgetcallbacknodex(CPXCENVptr env, void * cbdata, int wherefrom, double * x, 
		int begin, int end);
\end{lstlisting}
\end{tabular}
\end{center}
\begin{table}[h]
\centering
\begin{tabular}{rl}
\textbf{env:} & {puntatore una struttura ENV}\\
\textbf{cbdata:} & {puntatore che contiene specifiche informazioni per la callback}\\
\textbf{wherefrom:} & {contiene in che punto dell'ottimizzazione è stata invocata la callback} \\ 
\textbf{x:} & {vettore in cui memorizzare le variabili} \\
\textbf{begin} & {indice della prima variabile che si vuole venga restituita}\\
\textbf{end} & {indice dell'ultima variabile che si vuole venga restituita}\\  
\textbf{Return Value:} & {0 in caso di successo, un valore diverso da 0 se si verifica un errore}\\                               
\end{tabular}
\end{table}
Invece, per ottenere informazioni riguardanti il problema di ottimizzazione che si sta risolvendo all'interno di una callback implementata dall'utente, è possibile utilizzare:
\begin{center}
\begin{tabular}{c}
\begin{lstlisting}[linewidth=415pt, basicstyle=\footnotesize\sffamily,]  
int CPXgetcallbackinfo(CPXCENVptr env, void * cbdata, int wherefrom, int whichinfo, 
		void * result_p);
\end{lstlisting}
\end{tabular}
\end{center}
\begin{table}[h]
\centering
\begin{tabular}{rl}
\textbf{env:} & {puntatore ad una struttura ENV}\\
\textbf{cbdata:} & {puntatore che contiene specifiche informazioni per la callback}\\
\textbf{wherefrom:} & {contiene in che punto dell'ottimizzazione è stata invocata la callback}\\ 
\textbf{whichinfo:} & {macro che specifica l'informazione che si desidera conoscere} \\
\textbf{result\_p:} & {puntatore di tipo void in cui verrà memorizzata l'informazione richiesta}\\ 
\textbf{Return Value:} & {0 in caso di successo, un valore diverso da 0 se si verifica un errore}\\          
\end{tabular}
\end{table}
Macro utili da utilizzare come parametro \textit{whichinfo} possono essere:
\begin{table}[h]
\centering \footnotesize
\begin{tabular}{|r|l|}
\hline
\textbf{CPX\_CALLBACK\_INFO\_MY\_THREAD\_NUM:} & {identifica il thread che }\\
&{ha eseguito la chiamata}\\
\hline
\textbf{CPX\_CALLBACK\_INFO\_BEST\_INTEGER:} & {valore della miglior}\\
&{soluzione intera}\\
\hline
\end{tabular}
\end{table}
\\Per conoscere il valore della funzione obiettivo del problema legato al nodo corrente che invoca la callback:
\begin{center}
\begin{tabular}{c}
\begin{lstlisting}[linewidth=380pt, basicstyle=\footnotesize\sffamily,] 
int CPXgetcallbacknodeobjval(CPXCENVptr env, void * cbdata, int wherefrom, 
		double * objval_p); 
\end{lstlisting}
\end{tabular}
\end{center}
\begin{table}[H]
\centering
\begin{tabular}{rl}
\textbf{env:} & {puntatore ad una struttura ENV}\\
\textbf{cbdata:} & {puntatore che contiene specifiche informazioni per la callback}\\
\textbf{wherefrom:} & {contiene in che punto dell'ottimizzazione è stata invocata la callback} \\ 
\textbf{objval\_p:} & {puntatore ad una variabile in cui memorizzare il costo} \\
\textbf{Return Value:} & {0 in caso di successo, un valore diverso da 0 se si verifica un errore}\\  
\end{tabular}
\end{table}
All'interno della lazy callback è necessario aggiungere il taglio voluto al nodo corrente che la invoca. Questo può essere fatto in due diverse modalità: globale o locale.\\
Nel primo caso il vincolo aggiunto sarà visibile da tutti i nodi. Inoltre, in caso non lo ritenga più necessario, CPLEX potrà eliminarlo dal modello. Quest'operazione viene detta \textit{purge} e si verifica, ad esempio, quando un taglio non viene applicato per molte iterazioni consecutive. Per un vincolo globale viene chiamata la seguente funzione, che ne aggiunge uno alla volta:
\begin{center}
\begin{tabular}{c}
\begin{lstlisting}[linewidth=400pt, basicstyle=\footnotesize\sffamily,]  
int CPXcutcallbackadd( CPXCENVptr env, void * cbdata, int wherefrom, int nzcnt, 
		double rhs, int sense, int const * cutind, double const * cutval, 
		int purgeable );
\end{lstlisting}
\end{tabular}
\end{center}
\begin{table}[H]
\centering
\begin{tabular}{rl}
\textbf{env:} & {puntatore ad una struttura ENV}\\
\textbf{cbdata:} & {puntatore che contiene specifiche informazioni per la callback}\\
\textbf{wherefrom:} & {contiene in che punto dell'ottimizzazione è stata invocata la callback} \\ 
\textbf{nzcnt:} & {numero di coefficienti non nulli} \\
\textbf{rhs:} & {valore del termine noto} \\
\textbf{sense:} & {tipologia del taglio da aggiungere, a scelta tra} \\
&{\textit{'L'} per vincolo $\leq$}\\
&{\textit{'E'} per vincolo $=$}\\
&{\textit{'G'} per vincolo $\geq$}\\
\textbf{cutind:} & {vettore contente gli indici dei coefficienti del taglio} \\
\textbf{cutval:} & {vettore contenente i coefficienti delle variabili nel taglio} \\
\textbf{purgeable:} & {intero che specifica in che modo CPLEX deve trattare il taglio, consigliato 0} \\
\textbf{Return Value:} & {0 in caso di successo, un valore diverso da 0 se si verifica un errore}\\  
\end{tabular}
\end{table}
Nella seconda modalità, locale, il taglio aggiunto sarà visibile solo ai nodi discendenti di quello che invoca la callback. Viene implementata con la seguente chiamata:
\begin{center}
\begin{tabular}{c}
\begin{lstlisting}[linewidth=400pt, basicstyle=\footnotesize\sffamily,] 
int CPXcutcallbackaddlocal( CPXCENVptrenv, void *cbdata, int wherefrom, 
		int nzcnt, double rhs, int sense, int const *cutind, double const *cutval ); 
\end{lstlisting}
\end{tabular}
\end{center}
\begin{table}[h]
\centering
\begin{tabular}{rl}
\textbf{env:} & {puntatore ad una struttura ENV}\\
\textbf{cbdata:} & {puntatore che contiene specifiche informazioni per la callback}\\
\textbf{wherefrom:} & {contiene in che punto dell'ottimizzazione è stata invocata la callback} \\ 
\textbf{nzcnt:} & {numero di coefficienti non nulli} \\
\textbf{rhs:} & {valore del termine noto} \\
\textbf{sense:} & {tipologia del taglio da aggiungere, a scelta tra} \\
&{\textit{'L'} per vincolo $\leq$}\\
&{\textit{'E'} per vincolo $=$}\\
&{\textit{'G'} per vincolo $\geq$}\\
\textbf{cutind:} & {vettore contente gli indici dei coefficienti del taglio} \\
\textbf{cutval:} & {vettore contenente i coefficienti delle variabili nel taglio} \\
\textbf{Return Value:} & {0 in caso di successo, un valore diverso da 0 se si verifica un errore}\\ 
\end{tabular}
\end{table}
\subsection{Heuristic Callback}
Per poter suggerire a CPLEX una soluzione del problema in esame calcolata dall'utente, è necessario utilizzare un particolare tipo di callback, detta \textit{heuristic callback}. Questa, dopo essere stata installata, verrà invocata ad ogni nodo dell'albero del branch and cut.\\
Per installare la callback viene utilizzata la seguente funzione:
\begin{center}
\begin{tabular}{c}
\begin{lstlisting}[linewidth=330pt, basicstyle=\footnotesize\sffamily,]    
int CPXsetheuristiccallbackfunc(CPXENVptr env,
		 int(CPXPUBLIC *heuristiccallback)(CALLBACK_HEURISTIC_ARGS), 
		 void * cbhandle);
\end{lstlisting}
\end{tabular}
\end{center}
\begin{table}[h]
\centering
\begin{tabular}{rl}
\textbf{env:} & {puntatore ad una struttura ENV}\\
\textbf{heuristiccallback:} & {puntatore all'heuristic callback scritta dall'utente}\\
\textbf{cbhandle:} & {puntatore a dati privati dell'utente}\\
\textbf{Return Value:} & {0 in caso di successo, un valore diverso da 0 se si verifica un errore}\\
\end{tabular}
\end{table}
La callback dell'utente deve avere la dichiarazione specificata di seguito:
\begin{center}
\begin{tabular}{c}
\begin{lstlisting}[linewidth=382pt, basicstyle=\footnotesize\sffamily,] 
 int callback (CPXCENVptr env, void *cbdata, int wherefrom, void *cbhandle, 
 		double *objval_p, double *x, int *checkfeas_p, int *useraction_p);
\end{lstlisting}
\end{tabular}
\end{center}
\begin{table}[h]
\centering
\begin{tabular}{rl}
\textbf{env:} & {puntatore ad una struttura ENV}\\
\textbf{cbdata:} & {puntatore che contiene specifiche informazioni per la callback}\\
\textbf{wherefrom:} & {contiene in che punto dell'ottimizzazione è stata invocata la callback} \\ 
\textbf{cbhandle:} & {puntatore a dati privati dell'utente}\\
\textbf{objval\_p:} & {puntatore ad una variabile che in ingresso contiene il valore della funzione}\\
&{ obiettivo del problema e in uscita il valore della funzione obiettivo trovata}\\
&{nella funzione stessa, se esiste}\\
\textbf{x:} & {vettore che in ingresso contiene una soluzione valida per il problema }\\
&{e in uscita i valori della soluzione trovata nella funzione, se presente}\\
\textbf{checkfeas\_p:} & {puntatore che specifica se CPLEX deve verificare la soluzione trovata}\\
&{oppure no}\\
\textbf{useraction\_p:} & {puntatore ad un intero che specifica a CPLEX come proseguire la }\\
&{computazione al termine della callback dell'utente, scelto tra:}\\
&{\textit{CPX\_CALLBACK\_DEFAULT}: nessuna soluzione trovata}\\
&{\textit{CPX\_CALLBACK\_FAIL}: uscire dall'ottimizzazione}\\
&{\textit{CPX\_CALLBACK\_SET}: usare la soluzione fornita dall'utente}\\
\textbf{Return Value:} & {0 in caso di successo, un valore diverso da 0 se si verifica un errore}\\
\end{tabular}
\end{table}

\subsection{Generic Callback}% cambiare titolo da Lazy Constraint Callback General a general callback
Per evitare che alcune procedure interne a CPLEX vengano disattivate nel momento dell'installazione di una callback, recentemente ne è stata sviluppata una particolare tipologia, detta \textit{generic}. Questa non è relativa a una specifica versione di callback, come quelle descritte nelle sezioni precedenti, ma può essere invocata in contesti diversi e con diversi scopi.\\
Per installare una generic callback viene utilizzata la seguente funzione, in cui  è necessario specificare il contesto in cui invocare la callback:
\vspace{1 cm}
\begin{center}
\begin{tabular}{c}
\begin{lstlisting}[linewidth=380pt, basicstyle=\footnotesize\sffamily,]    
int CPXcallbacksetfunc ( CPXENVptr env, CPXLPptr lp, CPXLONG contextmask, 
		CPXCALLBACKFUNC *callback, void * userhandle );
\end{lstlisting}
\end{tabular}
\end{center}
\begin{table}[h]
\centering
\begin{tabular}{rl}
\textbf{env:} & {puntatore ad una struttura ENV}\\
\textbf{lp:} & {puntatore alla struttura LP}\\
\textbf{contextmask:} & {specifica in quali contesti deve essere invocata la callback, è possibile }\\
&{metterne in or anche più di uno e gestire poi i singoli casi dall'interno}\\
&{della funzione}\\
\textbf{callback:} & {puntatore alla callback scritta dall'utente} \\
\textbf{userhandle:} & {puntatore ad una struttura che contiene i dati da passare alla callback} \\
\textbf{Return Value:} & {0 in caso di successo, un valore diverso da 0 se si verifica un errore}\\
\end{tabular}
\end{table}
Il parametro \textit{contextmask} può variare a seconda dello scopo della callback creata. Alcuni possibili valori sono:
\begin{itemize}
\item{\textbf{CPX\_CALLBACKCONTEXT\_CANDIDATE}:\\la callback verrà invocata quando viene trovata da CPLEX una nuova soluzione possibile che l'utente potrà rifiutare;}
\item{\textbf{CPX\_CALLBACKCONTEXT\_LOCAL\_PROGRESS}: \\la callback verrà invocata nel momento in cui un thread effettua un progresso, non ancora noto globalmente, nella soluzione del problema. In questo contesto l'utente può suggerire a CPLEX una soluzione da cui proseguire la computazione (analogamente alle heuristic callback).}\\
\end{itemize}
L'utente può specificare più contesti con una sola installazione, è sufficiente separare le macro desiderate con l'operatore or bitwise ('|').\\
La user-callback implementata deve avere questa dichiarazione:
\begin{center}
\begin{tabular}{c}
\begin{lstlisting}[linewidth=350pt, basicstyle=\footnotesize\sffamily,]    
static int CPXPUBLIC name_general_callback(CPXCALLBACKCONTEXTptr
                     context, CPXLONG contextid, void* userhandle);
\end{lstlisting}
\end{tabular}
\end{center}
\begin{table}[h]
\centering
\begin{tabular}{rl}
\textbf{contex:} & {puntatore ad una struttura di contesto della callback}\\
%opaque callback context structure, verificare traduzione
\textbf{contextid:} & {intero che specifica il contesto in cui viene invocata la callback}\\
\textbf{userhandle:} & {argomento passato alla callback nell'installazione}\\
\textbf{Return Value:} & {0 in caso di successo, un valore diverso da 0 se si verifica un errore}\\
\end{tabular}
\end{table}
L'utente può installare una sola user-callback, ma al suo interno può distinguere il contesto in cui è stata invocata grazie al parametro \textit{contextid}.\\
Per poter accedere alla soluzione candidata e al suo costo, deve essere presente la seguente chiamata, che è specifica per il contesto \textbf{CPX\_CALLBACKCONTEXT\_CANDIDATE}:
\begin{center}
\begin{tabular}{c}
\begin{lstlisting}[linewidth=380pt, basicstyle=\footnotesize\sffamily,]
int CPXcallbackgetcandidatepoint( CPXCALLBACKCONTEXptr context, double *x, 
		CPXDIM begin, CPXDIM end, double *obj_p, );    
\end{lstlisting}
\end{tabular}
\end{center}
\begin{table}[h]
\centering
\begin{tabular}{rl}
\textbf{contex:} & {contesto, come passato alla callback scritta dall'utente}\\
\textbf{x:} & {vettore dove memorizzare i valori richiesti}\\
\textbf{begin:} & {indice prima colonna richiesta}\\
\textbf{end:} & {indice dell'ultima colonna richiesta}\\
\end{tabular}
\end{table}
\begin{table}[h]
\centering
\begin{tabular}{rl}
\textbf{obi\_p:} & {buffer in cui memorizzare il costo della soluzione candidata,}\\
&{può essere NULL}\\
\textbf{Return Value:} & {0 in caso di successo, un valore diverso da 0 se si verifica un errore}\\
\end{tabular}
\end{table}
Per poter scartare una soluzione, nel caso in cui violi alcuni tagli specificati nella chiamata stessa, viene utilizzata la seguente funzione. Anche questa è specifica per il contesto 
\\\textbf{CPX\_CALLBACKCONTEXT\_CANDIDATE}.
\begin{center}
\begin{tabular}{c}
\begin{lstlisting}[linewidth=420pt, basicstyle=\footnotesize\sffamily,] 
int CPXcallbackrejectcandidate( CPXCALLBACKCONTEXTptr context, int rcnt, int nzcnt, 
		double const *rhs, char const *sense, int const *rmatbeg, int const *rmatind, 
		double const *rmatval );   
\end{lstlisting}
\end{tabular}
\end{center}
\begin{table}[h]
\centering
\begin{tabular}{rl}
\textbf{contex:} & {contesto, come passato alla callback scritta dall'utente}\\
\textbf{rcnt:} & {numero di vincoli che tagliano la soluzione}\\
\textbf{nzcnt:} & {numero di coefficienti non nulli nel vincolo}\\
\textbf{rhs:} & {vettore di termini noti}\\
\textbf{sense:} & {vettore con la tipologia dei vincoli specificati}\\
\textbf{rmatbeg:} & {vettore di indici che specifica dove inizia ogni vincolo}\\
\textbf{rmatind:} & {vettore di indici delle colonne con coefficienti non nulli}\\
\textbf{rmatval:} & {coefficienti non nulli delle colonne specificate}\\
\textbf{Return Value:} & {0 in caso di successo, un valore diverso da 0 se si verifica un errore}\\
\end{tabular}
\end{table}
Per suggerire a CPLEX la soluzione da cui proseguire nella computazione dev'essere utilizzata la funzione:
\begin{center}
\begin{tabular}{c}
\begin{lstlisting}[linewidth=365pt, basicstyle=\footnotesize\sffamily,]
int CPXcallbackpostheursoln( CPXCALLBACKCONTEXTptr context, CPXDIM cnt, 
			CPXDIM const * ind, double const * val, double obj, 
			CPXCALLBACKSOLUTIONSTRATEGY strat );    
\end{lstlisting}
\end{tabular}
\end{center}
\begin{table}[h]
\centering
\begin{tabular}{rl}
\textbf{contex:} & {contesto, come passato alla callback scritta dall'utente}\\
\textbf{cnt:} & {numero di elementi nei vettori ind e val}\\
\textbf{ind:} & {vettore di indici non nulli dei valori della soluzione}\\
\textbf{val:} & {vettore di valori non nulli della soluzione, possono essere NaN nel caso in cui }\\
&{la soluzione sia parziale}\\
\textbf{obj:} & {costo della nuova soluzione}\\
\textbf{strat:} & {strategia con cui CPLEX deve completare la nuova soluzione, nel caso sia }\\
&{parziale, scelta tra:}\\
&{\textit{CPXCALLBACKSOLUTION\_NOCHECK} affinchè CPLEX}\\ 
&{non controlli l'attuabilità della soluzione (che deve essere completa) }\\
&{\textit{CPXCALLBACKSOLUTION\_CHECKFEAS} affinchè CPLEX}\\
&{controlli solamente se la soluzione è attuabile (la soluzione proposta }\\
&{deve essere completa)}\\
&{\textit{CPXCALLBACKSOLUTION\_PROPAGATE} affinchè CPLEX }\\
\end{tabular}
\end{table}
\begin{table}[h]
\centering
\begin{tabular}{rl}
&{cerchi di completare la soluzione attraverso la propagazione del bound}\\
&{\textit{CPXCALLBACKSOLUTION\_SOLVE} affinchè CPLEX}\\ 
&{fissi le variabili specificate nella soluzione e cerchi di risolvere il risultante }\\
&{problema ridotto}\\
\textbf{Return Value:} & {0 in caso di successo, un valore diverso da 0 se si verifica un errore}\\
\end{tabular}
\end{table}
CPLEX utilizzerà la soluzione proposta dall'utente solo nel caso in cui questa abbia costo inferiore all'incumbent.
\section{Parametri}\label{param}
Con le seguenti funzioni è possibile modificare i parametri di configurazione di CPLEX, altrimenti impostati ai valori di default.
Nel caso in cui si tratti di parametri di tipo INT è necessario invocare:\\
\begin{center}
\begin{tabular}{c}
\begin{lstlisting}[linewidth=330pt, basicstyle=\footnotesize\sffamily,]     
int CPXsetintparam(CPXENVptr env, int whichparam, int newvalue);
\end{lstlisting}
\end{tabular}
\end{center}
mentre se di tipo DOUBLE:\\
\begin{center}
\begin{tabular}{c}
\begin{lstlisting}[linewidth=340pt, basicstyle=\footnotesize\sffamily,]  
int CPXsetdblparam(CPXENVptr env, int whichparam, double newvalue);
\end{lstlisting}
\end{tabular}
\end{center}
In entrambe le funzioni:
\begin{table}[h]
\centering
\begin{tabular}{rl}
\textbf{env:} & {puntatore alla struttura ENV di cui si vogliono cambiare i parametri}\\
\textbf{whichparam:} & {intero corrispondente al parametro da modificare (vedi Tabella \ref{param_table})}\\
\textbf{newvalue:} & {nuovo valore (rispettivamente intero o double) del parametro}\\
\textbf{Return Value:} & {0 in caso di successo, un valore diverso da 0 se si verifica un errore}\\
\end{tabular}
\end{table}
\begin{table}[h]
\centering\footnotesize
\begin{tabular}{|l|l|}
\hline
\multirow{2}{*}{\textbf{CPX\_PARAM\_EPGAP}} & {tolleranza dell'intervallo tra la migliore funzione obiettivo intera}\\
& { e la funzione obiettivo del miglior nodo rimanente.}\\
\hline
\multirow{3}{*}{\textbf{CPX\_PARAM\_NODELIM}} & {massimo numero di nodi da risolvere prima che l'algoritmo}\\
& { termini senza aver aggiunto l'ottimalità}\\
& {(0 impone di fermarsi alla radice).}\\
\hline
\multirow{2}{*}{\textbf{CPX\_PARAM\_POPULATELIM}} & {limita il numero di soluzioni MIP generate per il pool  }\\
& {di soluzioni durante ogni chiamata alla procedura populate.}\\
\hline
\textbf{CPX\_PARAM\_SCRIND} & {visione o meno dei messaggi di log di CPLEX.}\\
\hline
\textbf{CPX\_PARAM\_MIPCBREDLP} & {permette, dalla callback chiamata, di accedere  }\\
&{al modello originale del problema e non a quello ridotto .}\\
\hline
\textbf{CPX\_PARAM\_THREADS} & {imposta il numero massimo di thread utilizzabili. }\\
\hline
\textbf{CPX\_PARAM\_RINSHEUR} & {imposta la frequenza (ogni quanti nodi) con cui deve}\\
&{essere invocato da CPLEX l'algoritmo euristico Rins.}\\
\hline
\textbf{CPX\_PARAM\_POLISHTIME} & {imposta quanto tempo in secondi deve dedicare CPLEX}\\
&{a fare il polish della soluzione.}\\
\hline
\end{tabular}
\end{table}
\begin{table}[h]
\centering\footnotesize
\begin{tabular}{|l|l|}
\hline
\textbf{CPX\_PARAM\_INTSOLLIM}&{imposta il numero di soluzioni MIP da trovare prima di fermarsi.}\\
\hline
\textbf{CPX\_PARAM\_TIMELIMIT}&{imposta il tempo massimo per il calcolo della soluzione.}\\
\hline
\textbf{CPX\_PARAM\_RANDOMSEED}&{imposta il random seed.}\\
\hline
\end{tabular}
\caption{Parametri.}\label{param_table}
\end{table}
\vspace{2 cm}
\section{Costanti utili}
Di seguito sono riportate alcune macro utili di CPLEX, insieme ai loro corrispondenti valori:
\begin{table}[h]
\footnotesize\centering
\begin{tabular}{|r|l|}
\hline
\textbf{CPX\_ON} & {\textbf{1}}\\
{} & {valore da assegnare ad alcuni parametri per abilitarli}\\
\hline
\textbf{CPX\_OFF} & {0}\\
{} & {valore da assegnare ad alcuni parametri per disabilitarli}\\
\hline
\textbf{CPX\_INFBOUND} & {$+\infty$}\\
{} & {massimo valore intero utilizzabile in CPLEX}\\
\hline
\end{tabular}
\end{table}

\chapter{Gnuplot}\label{gnuplot}
Nella nostra implementazione, una volta ottenuta la soluzione del problema di ottimizzazione, ne viene disegnato il grafo per facilitare all'utente la comprensione della sua correttezza. Per fare ciò viene utilizzato GNUplot, un programma di tipo command-driven.\\
Per poterlo sfruttare all'interno del proprio programma esistono due metodi:
\begin{itemize}
\item{Collegare la libreria ed invocare le sue funzioni all'interno del programma}
\item{Collegare l'eseguibile interattivo al proprio programma. In questo caso i comandi devono essere passati all'eseguibile attraverso l'utilizzo di un file di testo e di un pipe.}\\
\end{itemize}
In questa trattazione è stato scelto il secondo metodo. All'interno di un file vengono specificati i comandi da eseguire in Gnuplot e le caratteristiche grafiche che deve aver il grafo da rappresentare. Un esempio di tale file, viene riportato nelle seguenti righe:\\

\lstinputlisting[caption={\footnotesize{style.txt}}, style=code, firstnumber=1, firstline=1, lastline=12, label=style_example]{Source/style_example.txt}

Nell'esempio citato, nella prima parte viene definito lo stile, il colore delle linee e la tipologia di punti, che verrano in seguito visualizzati all'interno del grafico prodotto.\\ In seguito viene effettuato il plot, analizzando il file \textbf{solution.dat}, contenente le informazioni relative alla soluzione del problema, in cui ciascuna riga ha la seguente forma:
\begin{center}
\begin{tabular}{c}
\begin{lstlisting}[linewidth=290pt,basicstyle=\footnotesize\sffamily,]     
coordinata_x   coordinata_y   posizione_nel_tour
\end{lstlisting}
\end{tabular}
\end{center}
\textbf{coordinata\_x} rappresenta la coordinata x del nodo;\\
\textbf{coordinata\_y} rappresenta la coordinata y del nodo;\\
\textbf{posizione\_nel\_tour} rappresenta l'ordine del nodo all'interno del tour, assunto come nodo di origine il nodo 1.\\\\
Il grafico viene generato dal comando \textbf{plot}, leggendo tutte le righe non vuote e disegnando un punto nella posizione \textbf{(coordinata\_x, coordinata\_y)} del grafico 2D. In seguito viene tracciata una linea solo tra coppie di punti legati a righe consecutive non vuote nel file \textbf{solution.dat}.\\\\
Attraverso le istruzioni riportate nelle righe 10-12 di \textbf{style.txt}, viene invece salvato il grafico appena generato nell'immagine \textbf{solution.png}.\\\\
Di seguito vengono riportate le varie fasi necessarie alla definizione di un pipe e all'utilizzo di questo per eseguire comandi in GNUplot:
\begin{itemize}
\item{\textbf{Definizione del pipe}
\lstinputlisting[style=code, firstnumber=1, firstline=1, lastline=1, label=style_example language=C]{Source/gnuplotC.txt}
dove \textbf{GNUPLOT\_EXE} è una stringa composta dal percorso completo dell'eseguibile di GNUplot, seguita dall'argomento \textbf{-persistent} (es. \textit{"D:/Programs/GNUplot/bin/gnuplot.exe -persistent"}).
}
\item{\textbf{Passaggio delle istruzioni a GNUplot}
\lstinputlisting[style=code, firstnumber=2, firstline=2, lastline=10, label=style_example, language=C]{Source/gnuplotC.txt}
viene passata una riga alla volta, del file \textbf{style.txt}, a GNUplot mediante il pipe precedentemente creato.
}
\item{\textbf{Chiusura del pipe}
\lstinputlisting[style=code, firstnumber=11, firstline=11, lastline=11, label=style_example, language=C]{Source/gnuplotC.txt}
}
\end{itemize}
\chapter{Performance profile in python}\label{perf_profile_py}
Il programma utilizzato per la creazione del performance profile dei diversi algoritmo è perprof.py\cite{salvagnin_perf}. Di seuito vengono riportati i principali argomenti da linea di comando che possono essere utilizzati:

\begin{table}[h]
\begin{tabular}{|r|l|}
\hline
\textbf{-D delimiter} & {spefica che delimiter verrà usato come separatore tra le}\\
& {parole in una riga}\\
\hline
\textbf{-M value} & {imposta value come il massimo valore di ratio (asse x)}\\
\hline
\textbf{-S value} & {value rappresenta la quantità che viene sommata a}\\
& {ciascun tempo di esecuzione prima del confronto.}\\
& {Questo parametro è utile per non enfatizzare troppo}\\
& {le differenze di pochi ms tra gli algoritmi.}\\
\hline
\textbf{-L} & {stampa in scala logaritmica}\\
\hline
\textbf{-T value} & {nel file passato al programma, il TIME LIMIT=value}\\
\hline
\textbf{-P "title"} & {title è il titolo del plot}\\
\hline
\textbf{-X value} & {nome dell'asse x (default='Time Ratio')}\\
\hline
\textbf{-B} & {plot in bianco e nero}\\
\hline
\end{tabular}
\end{table}
Di seguito viene riportato un esempio dell'esecuzione del programma, del suo input e del suo output:
\begin{itemize}
\item{\textbf{comando}
\begin{center}
\begin{tabular}{c}
\begin{lstlisting}[linewidth=330pt, basicstyle=\footnotesize\sffamily,] 
python perfprof.py -D , -T 3600 -S 2 -M 20 esempio.csv
             out.pdf -P "all instances, shift 2 sec.s"
\end{lstlisting}
\end{tabular}
\end{center}
}
\item{\textbf{file di input con i dati}\\
Viene riportato parte del contenuto di esempio.csv .
\begin{center}
\begin{tabular}{c}
\begin{lstlisting}[linewidth=240pt, basicstyle=\footnotesize\sffamily,] 
3, Alg1, Alg2, Alg3
model_1.lp, 2.696693, 3.272468, 2.434147
model_2.lp, 0.407689, 1.631921, 1.198957
model_3.lp, 0.333669, 0.432553, 0.966638
\end{lstlisting}
\end{tabular}
\end{center}
La prima riga deve necessariamente contenere in ordine il numero di algoritmi analizzati e i loro nomi. Nelle righe seguenti viene riportato invece il nome del file lp e i tempi di esecuzione elencati secondo la sequenza di algoritmi specificata all'inizio.
Ogni campo di ciascuna riga deve essere separato dal delimitatore specificato all'avvio del programma attraverso l'opzione -D.
}
\item{\textbf{immagine di output}\\
Il grafico viene restituito nel file out.pdf specificato da line di comando chiamando il programma.
\begin{figure}[h] 
\begin{center} 
  % Requires \usepackage{graphicx} 
  \includegraphics[scale=0.6]{Images/profile_out}\\ 
\end{center} 
\end{figure}
}
\end{itemize}
\chapter{Risultati}\label{results}
{\footnotesize
\begin{longtable}[H]{lrrr}
\caption{Tempo di esecuzione degli algoritmi compatti con time limit di 20 minuti.}\\
\hline
{} & \textbf{MTZ con} & {} & {}\\
{} & \textbf{lazy constraint (}s\textbf{)} & \textbf{MTZ (}s\textbf{)} & \textbf{GG (}s\textbf{)}\\
\hline
\textit{att48.tsp} & 186.679 & 405.781 & 12.532\\
\textit{berlin52.tsp} & 1.406 & 5.486 & 7.551\\
\textit{burma14.tsp} & 0.126 & 0.271 & 0.201\\
\textit{eil101.tsp} & 6.136 & 268.616 & 228.185\\
\textit{eil51.tsp} & 2.265 & 5.302 & 31.927\\
\textit{eil76.tsp} & 2.615 & 30.064 & 96.272\\
\textit{gr96.tsp} & 731.232 & 575.945 & 371.457\\
\textit{kroA100.tsp} & TIME LIMIT & TIME LIMIT & 468.402\\
\textit{kroB100.tsp} & TIME LIMIT & TIME LIMIT & 638.129\\
\textit{kroB150.tsp} & TIME LIMIT & TIME LIMIT & TIME LIMIT\\
\textit{kroC100.tsp} & TIME LIMIT & TIME LIMIT & 299.991\\
\textit{kroD100.tsp} & TIME LIMIT & TIME LIMIT & 239.841\\
\textit{kroE100.tsp} & 1203.42 & TIME LIMIT & 306.457\\
\textit{pr124.tsp} & TIME LIMIT & TIME LIMIT & TIME LIMIT\\
\textit{pr136.tsp} & TIME LIMIT & TIME LIMIT & TIME LIMIT\\
\textit{pr76.tsp} & 425.937 & 1112.979 & 482.098\\
\textit{rat99.tsp} & 10.136 & 281.254 & 167.077\\
\textit{rd100.tsp} & 459.557 & 623.765 & 254.192\\
\textit{st70.tsp} & 260.757 & TIME LIMIT & 60.892\\
\textit{ulysses16.tsp} & 1.801 & 2.749 & 0.630\\
\hline
\end{longtable}
}

{\footnotesize
\begin{longtable}[H]{lrrrrr}
\caption{Tempo di esecuzione degli algoritmi esatti con time limit di 10 minuti.}\\
\hline
{} & {} & {} & {} & \textbf{B\&C Generic} & \textbf{B\&C +}\\
{} & \textbf{Loop (}s\textbf{)} & \textbf{B\&C Generic (}s\textbf{)} & \textbf{B\&C (}s\textbf{)} & \textbf{+ Patching (}s\textbf{)} & \textbf{Patching (}s\textbf{)}\\
\hline
\textit{a280.tsp} & 13.857 & 9.509 & 4.187 & 8.211 & 4.191\\
\textit{ali535.tsp} & 251.844 & 119.427 & TIME LIMIT & 314.189 & TIME LIMIT\\
\textit{att48.tsp} & 0.972 & 0.252 & 0.202 & 0.251 & 0.189\\
\textit{att532.tsp} & 313.906 & 376.933 & 600.842 & 567.807 & TIME LIMIT\\
\textit{berlin52.tsp} & 0.141 & 0.179 & 0.063 & 0.191 & 0.051\\
\textit{bier127.tsp} & 2.439 & 1.213 & 1.283 & 1.749 & 1.291\\
\textit{burma14.tsp} & 0.066 & 0.034 & 0.031 & 0.031 & 0.021\\
\textit{ch130.tsp} & 2.411 & 1.031 & 1.474 & 1.771 & 1.159\\
\textit{ch150.tsp} & 5.951 & 1.841 & 3.211 & 2.679 & 2.491\\
\textit{d198.tsp} & 26.099 & 6.578 & 39.322 & 11.712 & 13.501\\
\textit{d493.tsp} & 317.945 & 266.471 & TIME LIMIT & 179.002 & TIME LIMIT\\
\textit{d657.tsp} & 458.962 & TIME LIMIT & TIME LIMIT & TIME LIMIT & TIME LIMIT\\
\textit{eil51.tsp} & 0.749 & 0.172 & 0.167 & 0.156 & 0.121\\
\textit{eil76.tsp} & 0.285 & 0.185 & 0.094 & 0.212 & 0.113\\
\textit{eil101.tsp} & 1.081 & 0.261 & 0.271 & 0.354 & 0.269\\
\textit{fl417.tsp} & 209.508 & 306.274 & TIME LIMIT & 122.981 & TIME LIMIT\\
\textit{gil262.tsp} & 26.517 & 30.878 & 18.357 & 13.989 & 27.911\\
\textit{gr96.tsp} & 2.131 & 0.765 & 0.421 & 0.551 & 0.708\\
\textit{gr137.tsp} & 3.428 & 0.947 & 1.064 & 1.121 & 1.121\\
\textit{gr202.tsp} & 19.909 & 4.104 & 6.073 & 5.531 & 6.329\\
\textit{gr229.tsp} & 16.537 & 14.832 & 4.919 & 12.648 & 9.122\\
\textit{gr431.tsp} & 72.726 & 35.762 & 46.931 & 46.832 & 86.982\\
\textit{gr666.tsp} & TIME LIMIT & 190.031 & TIME LIMIT & TIME LIMIT & TIME LIMIT\\
\textit{kroA100.tsp} & 2.011 & 0.829 & 0.861 & 0.651 & 0.882\\
\textit{kroA150.tsp} & 10.519 & 2.603 & 2.796 & 3.412 & 2.271\\
\textit{kroA200.tsp} & 24.679 & 13.581 & 34.077 & 8.682 & 74.918\\
\textit{kroB100.tsp} & 4.225 & 1.552 & 0.869 & 1.271 & 0.659\\
\textit{kroB150.tsp} & 14.636 & 4.824 & 5.791 & 4.472 & 6.111\\
\textit{kroB200.tsp} & 12.244 & 7.161 & 6.088 & 10.427 & 8.239\\
\textit{kroC100.tsp} & 2.628 & 0.739 & 0.631 & 0.711 & 0.769\\
\textit{kroD100.tsp} & 2.113 & 0.496 & 0.832 & 0.673 & 0.709\\
\textit{kroE100.tsp} & 2.477 & 1.357 & 0.681 & 0.948 & 0.931\\
\textit{lin105.tsp} & 1.674 & 0.642 & 0.472 & 0.311 & 0.419\\
\textit{lin318.tsp} & 47.514 & 25.434 & 37.783 & 35.308 & 100.161\\
\textit{p654.tsp} & TIME LIMIT & 157.174 & 85.571 & TIME LIMIT & 308.872\\
\textit{pcb442.tsp} & 243.732 & 15.792 & 36.125 & 22.309 & 35.702\\
\textit{pr76.tsp} & 4.225 & 2.546 & 4.716 & 1.921 & 9.161\\
\textit{pr107.tsp} & 0.614 & 0.112 & 0.081 & 0.131 & 0.081\\
\textit{pr124.tsp} & 8.843 & 1.707 & 1.149 & 2.021 & 1.721\\
\textit{pr136.tsp} & 2.754 & 1.582 & 1.093 & 1.409 & 0.842\\
\textit{pr144.tsp} & 11.239 & 3.109 & 2.891 & 3.848 & 3.149\\
\textit{pr152.tsp} & 5.479 & 4.995 & 2.038 & 4.842 & 2.771\\
\textit{pr226.tsp} & 66.161 & 17.555 & 9.497 & 14.279 & 9.422\\
\textit{pr299.tsp} & 108.237 & 125.979 & TIME LIMIT & 51.142 & TIME LIMIT\\
\textit{pr439.tsp} & 409.73 & 501.036 & TIME LIMIT & 362.521 & TIME LIMIT\\
\textit{rat99.tsp} & 1.548 & 0.345 & 0.532 & 0.321 & 0.739\\
\textit{rat195.tsp} & 37.502 & 12.853 & 12.693 & 10.908 & 23.791\\
\textit{rat575.tsp} & 391.694 & 281.827 & TIME LIMIT & 352.712 & TIME LIMIT\\
\textit{rat783.tsp} & 222.909 & 381.214 & TIME LIMIT & TIME LIMIT & TIME LIMIT\\
\textit{rd100.tsp} & 2.306 & 1.122 & 0.573 & 0.684 & 0.968\\
\textit{rd400.tsp} & 128.746 & 80.335 & TIME LIMIT & 46.511 & 601.14\\
\textit{st70.tsp} & 0.466 & 0.193 & 0.217 & 0.232 & 0.221\\
\textit{u159.tsp} & 3.285 & 2.096 & 1.374 & 1.838 & 2.012\\
\textit{u574.tsp} & 157.913 & 161.905 & TIME LIMIT & 194.281 & TIME LIMIT\\
\textit{u724.tsp} & 440.595 & TIME LIMIT & TIME LIMIT & TIME LIMIT & TIME LIMIT\\
\textit{ulysses16.tsp} & 0.232 & 0.121 & 0.047 & 0.051 & 0.039\\
\textit{ulysses22.tsp} & 0.186 & 0.067 & 0.031 & 0.081 & 0.041\\
\hline
\end{longtable}
}

{\footnotesize
\begin{longtable}[H]{lrrr}
\caption{Tempo di esecuzione dell'algoritmo loop, con preprocessamento euristico, al variare del gap relativo e con time limit di 10 minuti.}\\
\cline{2-4}
{} & \multicolumn{3}{c}{\textbf{Gap}}\\
\hline
{} & \textbf{0.1} & \textbf{0.01} & \textbf{DEFAULT}\\
\hline
\textit{a280.tsp} & 27.411 & 11.588 & 13.857\\
\textit{ali535.tsp} & 291.323 & 349.553 & 251.844\\
\textit{att48.tsp} & 0.684 & 0.629 & 0.972\\
\textit{att532.tsp} & 281.506 & 306.981 & 313.906\\
\textit{berlin52.tsp} & 0.166 & 0.141 & 0.141\\
\textit{bier127.tsp} & 2.330 & 2.918 & 2.439\\
\textit{burma14.tsp} & 0.121 & 0.099 & 0.066\\
\textit{ch130.tsp} & 2.624 & 2.360 & 2.411\\
\textit{ch150.tsp} & 6.972 & 7.128 & 5.951\\
\textit{d198.tsp} & 19.695 & 20.837 & 26.099\\
\textit{d493.tsp} & 330.015 & 241.929 & 317.945\\
\textit{d657.tsp} & TIME LIMIT & 445.697 & 458.962\\
\textit{eil101.tsp} & 1.173 & 0.805 & 1.080\\
\textit{eil51.tsp} & 0.566 & 0.471 & 0.749\\
\textit{eil76.tsp} & 0.342 & 0.259 & 0.285\\
\textit{fl417.tsp} & 217.655 & 313.192 & 209.508\\
\textit{gil262.tsp} & 22.302 & 25.560 & 26.517\\
\textit{gr137.tsp} & 5.941 & 4.857 & 3.428\\
\textit{gr202.tsp} & 19.001 & 16.674 & 19.909\\
\textit{gr229.tsp} & 12.574 & 10.783 & 16.537\\
\textit{gr431.tsp} & 91.827 & 66.614 & 72.726\\
\textit{gr666.tsp} & 486.959 & 516.687 & TIME LIMIT\\
\textit{gr96.tsp} & 3.299 & 2.167 & 2.131\\
\textit{kroA100.tsp} & 2.245 & 3.097 & 2.011\\
\textit{kroA150.tsp} & 8.224 & 7.761 & 10.519\\
\textit{kroA200.tsp} & 31.376 & 29.324 & 24.679\\
\textit{kroB100.tsp} & 2.940 & 4.361 & 4.225\\
\textit{kroB150.tsp} & 12.204 & 13.437 & 14.636\\
\textit{kroB200.tsp} & 13.463 & 8.076 & 12.244\\
\textit{kroC100.tsp} & 2.561 & 2.196 & 2.628\\
\textit{kroD100.tsp} & 2.699 & 2.293 & 2.113\\
\textit{kroE100.tsp} & 3.065 & 1.997 & 2.477\\
\textit{lin105.tsp} & 1.54 & 1.829 & 1.674\\
\textit{lin318.tsp} & 56.505 & 49.786 & 47.514\\
\textit{p654.tsp} & TIME LIMIT & TIME LIMIT & TIME LIMIT\\
\textit{pcb442.tsp} & 251.024 & 266.387 & 243.732\\
\textit{pr107.tsp} & 0.551 & 0.555 & 0.614\\
\textit{pr124.tsp} & 8.394 & 7.002 & 8.840\\
\textit{pr136.tsp} & 4.676 & 2.741 & 2.750\\
\textit{pr144.tsp} & 5.316 & 11.016 & 11.239\\
\textit{pr152.tsp} & 4.719 & 4.863 & 5.480\\
\textit{pr226.tsp} & 32.678 & 43.278 & 66.161\\
\textit{pr299.tsp} & 102.936 & 109.369 & 108.237\\
\textit{pr439.tsp} & 355.795 & 476.181 & 409.73\\
\textit{pr76.tsp} & 5.331 & 4.867 & 4.225\\
\textit{rat195.tsp} & 32.627 & 36.468 & 37.502\\
\textit{rat575.tsp} & 305.992 & 417.810 & 391.694\\
\textit{rat783.tsp} & 310.511 & 283.290 & 222.909\\
\textit{rat99.tsp} & 2.178 & 1.420 & 1.548\\
\textit{rd100.tsp} & 1.932 & 1.900 & 2.306\\
\textit{rd400.tsp} & 110.792 & 133.082 & 128.746\\
\textit{st70.tsp} & 0.467 & 0.411 & 0.466\\
\textit{u159.tsp} & 2.439 & 3.145 & 3.285\\
\textit{u574.tsp} & 285.595 & 245.335 & 157.913\\
\textit{u724.tsp} & TIME LIMIT & 551.932 & 440.595\\
\textit{ulysses16.tsp} & 0.191 & 0.158 & 0.232\\
\textit{ulysses22.tsp} & 0.176 & 0.146 & 0.186\\
\hline
\end{longtable}
}

{\footnotesize
\begin{longtable}[H]{lrrrr}
\caption{Tempo di esecuzione dell'algoritmo loop al variare del random seed e con time limit di 10 minuti.}\\
\cline{2-5}
{} & \multicolumn{4}{c}{\textbf{Seed}}\\
\hline
{} & \textbf{100} & \textbf{250} & \textbf{500} & \textbf{1000}\\
\hline
\textit{a280.tsp} & 16.556 & 17.365 & 14.718 & 16.114\\
\textit{ali535.tsp} & 237.82 & 244.416 & 204.061 & 268.006\\
\textit{att48.tsp} & 0.664 & 0.613 & 0.695 & 0.605\\
\textit{att532.tsp} & 337.888 & 349.073 & 306.43 & 317.942\\
\textit{berlin52.tsp} & 0.179 & 0.202 & 0.164 & 0.184\\
\textit{bier127.tsp} & 2.425 & 2.557 & 2.704 & 2.63\\
\textit{burma14.tsp} & 0.065 & 0.121 & 0.066 & 0.06\\
\textit{ch130.tsp} & 2.444 & 2.519 & 2.636 & 2.582\\
\textit{ch150.tsp} & 5.216 & 5.228 & 5.446 & 5.286\\
\textit{d493.tsp} & 298.674 & 256.219 & 268.271 & 292.941\\
\textit{d657.tsp} & 529.713 & 476.444 & 511.532 & 443.609\\
\textit{eil51.tsp} & 0.497 & 0.505 & 0.574 & 0.614\\
\textit{eil101.tsp} & 0.973 & 1.003 & 1.476 & 1.005\\
\textit{eil76.tsp} & 0.291 & 0.288 & 0.287 & 0.346\\
\textit{fl417.tsp} & 210.629 & 216.737 & 228.297 & 237.123\\
\textit{gil262.tsp} & 25.446 & 21.953 & 23.957 & 25.757\\
\textit{gr96.tsp} & 2.361 & 2.354 & 2.35 & 2.351\\
\textit{gr137.tsp} & 3.328 & 3.512 & 3.4 & 3.712\\
\textit{gr202.tsp} & 19.77 & 20.923 & 18.47 & 20.154\\
\textit{gr229.tsp} & 18.043 & 18.269 & 17.369 & 17.273\\
\textit{gr431.tsp} & 64.513 & 71.89 & 63.802 & 76.604\\
\textit{gr666.tsp} & TIME LIMIT & TIME LIMIT & TIME LIMIT & TIME LIMIT\\
\textit{kroA100.tsp} & 2.198 & 2.468 & 2.154 & 2.263\\
\textit{kroA150.tsp} & 10.741 & 11.431 & 9.918 & 11.854\\
\textit{kroA200.tsp} & 26.872 & 23.063 & 24.315 & 31.149\\
\textit{kroB100.tsp} & 4.229 & 4.456 & 4.568 & 4.329\\
\textit{kroB150.tsp} & 14.127 & 13.261 & 14.596 & 15.419\\
\textit{kroB200.tsp} & 11.209 & 10.976 & 11.562 & 14.202\\
\textit{kroC100.tsp} & 2.441 & 2.694 & 2.361 & 2.578\\
\textit{kroD100.tsp} & 2.282 & 2.33 & 2.384 & 2.289\\
\textit{kroE100.tsp} & 2.224 & 2.204 & 2.555 & 2.306\\
\textit{lin105.tsp} & 1.623 & 1.608 & 1.596 & 1.783\\
\textit{lin318.tsp} & 43.449 & 50.029 & 47.338 & 54.559\\
\textit{p654.tsp} & TIME LIMIT & TIME LIMIT & TIME LIMIT & TIME LIMIT\\
\textit{pcb442.tsp} & 232.039 & 232.068 & 238.458 & 233.961\\
\textit{pr76.tsp} & 4.827 & 4.587 & 4.213 & 4.625\\
\textit{pr107.tsp} & 0.488 & 0.481 & 0.587 & 0.502\\
\textit{pr124.tsp} & 9.863 & 11.128 & 10.569 & 10.933\\
\textit{pr136.tsp} & 2.71 & 2.653 & 2.412 & 2.897\\
\textit{pr144.tsp} & 11.542 & 11.505 & 10.922 & 11.568\\
\textit{pr152.tsp} & 6.568 & 5.744 & 5.657 & 6.087\\
\textit{pr299.tsp} & 94.165 & 98.643 & 97.54 & 90.642\\
\textit{pr439.tsp} & 437.816 & 434.683 & 463.188 & 459.081\\
\textit{rat195.tsp} & 39.787 & 39.102 & 42.18 & 40.973\\
\textit{rat575.tsp} & 403.654 & 385.722 & 402.831 & 426.853\\
\textit{rat783.tsp} & 230.542 & 254.016 & 214.378 & 218.696\\
\textit{rat99.tsp} & 1.571 & 1.547 & 1.569 & 1.52\\
\textit{rd100.tsp} & 2.189 & 2.086 & 2.119 & 2.037\\
\textit{rd400.tsp} & 112.157 & 114.066 & 115.761 & 108.751\\
\textit{st70.tsp} & 0.406 & 0.422 & 0.463 & 0.448\\
\textit{u159.tsp} & 3.771 & 3.376 & 3.579 & 3.489\\
\textit{u574.tsp} & 187.302 & 189.366 & 165.305 & 172.686\\
\textit{u724.tsp} & 454.542 & 423.676 & 542.278 & 428.793\\
\textit{ulysses16.tsp} & 0.164 & 0.214 & 0.137 & 0.149\\
\textit{ulysses22.tsp} & 0.145 & 0.127 & 0.131 & 0.133\\
\hline
\end{longtable}
}

{\footnotesize
\begin{longtable}[H]{lrrrr}
\caption{Costo della soluzione ottenuta mediante algoritmi mat-euristici con time limit di 10 minuti.}\\
\hline
{} & \textbf{Soft Fixing} & \textbf{Hard Fixing} & \textbf{Generic Soft Fixing} & \textbf{Generic Hard Fixing}\\
\hline
\textit{a280.tsp} & 2724.77 & 2586.77 & 2608.26 & 2586.77\\
\textit{att532.tsp} & 1147110.24 & 94035.59 & 386595.10 & 87368.30\\
\textit{bier127.tsp} & 118293.52 & 118293.52 & 118293.52 & 118293.52\\
\textit{d198.tsp} & 15822.50 & 15808.65 & 15808.65 & 15808.65\\
\textit{d493.tsp} & 246364.95 & 35043.38 & 244907.83 & 35018.92\\
\textit{eil76.tsp} & 544.37 & 544.37 & 544.37 & 544.37\\
\textit{eil101.tsp} & 640.21 & 640.21 & 640.21 & 640.21\\
\textit{fl417.tsp} & 309353.68 & 13851.95 & 235146.98 & 11966.50\\
\textit{gr137.tsp} & 706.29 & 706.29 & 706.29 & 706.29\\
\textit{gr202.tsp} & 486.50 & 486.35 & 486.78 & 486.35\\
\textit{lin105.tsp} & 14383.00 & 14383.00 & 14383.00 & 14383.00\\
\textit{lin318.tsp} & 44649.54 & 42042.54 & 42258.85 & 42275.83\\
\textit{pcb442.tsp} & 50783.55 & 50783.55 & 177301.04 & 50783.55\\
\textit{pr144.tsp} & 58535.22 & 58535.22 & 58535.22 & 58535.22\\
\textit{pr299.tsp} & 48469.13 & 48323.36 & 48226.92 & 48194.92\\
\textit{pr439.tsp} & 107537.87 & 108632.65 & 501047.16 & 107332.47\\
\textit{rat575.tsp} & 84631.45 & 6934.26 & 7533.38 & 6799.55\\
\textit{rd400.tsp} & 15290.98 & 15285.50 & 23080.53 & 15275.98\\
\textit{u159.tsp} & 42075.67 & 42075.67 & 42075.67 & 42075.67\\
\hline
\end{longtable}
}

\vspace{1cm}
{\footnotesize
\begin{longtable}[H]{lrrrr}
\caption{Costo della soluzione ottenuta mediante Multistart, generando  40 possibili soluzioni in multithreading e restituendo la migliore di queste.}\\
\hline
{} & {} & \textbf{Insertion} & & \textbf{Nearest Neighborhod}\\
{} & \textbf{Insertion} & \textbf{+ GRASP} & \textbf{Nearest Neighborhod} & \textbf{+ GRASP}\\
\hline
\textit{a280.tsp} & 2817.92 & 2796.68 & 2772.51 & 2784.32\\
\textit{ali535.tsp} & 2186.59 & 2194.68 & 2125.32 & 2156.74\\
\textit{att532.tsp} & 95278.72 & 95278.72 & 92197.86 & 92571.30\\
\textit{d1291.tsp} & 55837.31 & 55837.31 & 54471.31 & 55255.14\\
\textit{d1665.tsp} & 67388.14 & 67847.16 & 66719.25 & 67200.25\\
\textit{d2103.tsp} & 83354.07 & 83354.07 & 82445.24 & 84145.24\\
\textit{d493.tsp} & 37991.33 & 37937.91 & 36748.84 & 37047.52\\
\textit{d657.tsp} & 54818.10 & 54524.77 & 51685.30 & 52341.20\\
\textit{dsj1000.tsp} & 20741407.35 & 20657356.07 & 20359263.48 & 20214411.67\\
\textit{fl1400.tsp} & 21580.66 & 21646.28 & 21291.85 & 21301.27\\
\textit{fl1577.tsp} & 24438.44 & 24405.16 & 23307.58 & 23392.2\\
\textit{fl417.tsp} & 12706.85 & 12712.88 & 12326.09 & 12278.76\\
\textit{gil262.tsp} & 2614.51 & 2590.11 & 2521.21 & 2535.53\\
\textit{gr431.tsp} & 2046.59 & 2046.59 & 2044.25 & 2060.65\\
\textit{gr666.tsp} & 3374.16 & 3377.25 & 3317.72 & 3317.72\\
\textit{lin318.tsp} & 45986.47 & 45926.30 & 45532.06 & 45480.76\\
\textit{nrw1379.tsp} & 62273.17 & 62295.46 & 60523.18 & 60843.52\\
\textit{p654.tsp} & 38113.48 & 38255.21 & 35381.28 & 35883.25\\
\textit{pcb1173.tsp} & 62190.34 & 62811.54 & 62436.66 & 62787.08\\
\textit{pcb442.tsp} & 55455.11 & 55414.92 & 53002.57 & 52439.6\\
\textit{pr1002.tsp} & 285476.34 & 284900.40 & 278473.49 & 280469.29\\
\textit{pr299.tsp} & 52751.57 & 52282.79 & 50920.49 & 52446.53\\
\textit{pr439.tsp} & 117315.31 & 117717.59 & 113483.70 & 115637.95\\
\textit{rat575.tsp} & 7421.49 & 7420.70 & 7184.38 & 7206.17\\
\textit{rat783.tsp} & 9670.53 & 9664.22 & 9418.27 & 9508.13\\
\textit{rd400.tsp} & 16649.07 & 16515.94 & 16307.31 & 16432.46\\
\textit{rl1304.tsp} & 282135.94 & 283443.11 & 274723.68 & 274723.68\\
\textit{rl1323.tsp} & 297641.77 & 301249.17 & 290628.52 & 291343.66\\
\textit{rl1889.tsp} & 352153.29 & 352256.86 & 339944.06 & 346590.70\\
\textit{u1060.tsp} & 251325.95 & 250829.17 & 241584.55 & 242756.36\\
\textit{u1432.tsp} & 168218.63 & 167766.50 & 165972.05 & 166245.09\\
\textit{u1817.tsp} & 63120.91 & 63120.91 & 62143.87 & 62035.52\\
\textit{u574.tsp} & 40965.17 & 41019.72 & 39122.03 & 39867.53\\
\textit{u724.tsp} & 47088.64 & 47271.16 & 44935.42 & 44921.80\\
\textit{vm1084.tsp} & 260985.29 & 261148.58 & 256049.81 & 260414.62\\
\textit{vm1748.tsp} & 375447.73 & 372341.92 & 358951.32 & 358951.32\\
\hline
\end{longtable}
}
\vspace{4cm}
{\footnotesize
\begin{longtable}[H]{lrrrrrr}
\caption{Costo della soluzione ottenuta mediante algoritmi euristici con time limit di 10 minuti.}\\
\hline
{} & {} & {} & \textbf{Hybrid VNS} & {} & \textbf{Simulated} & {}\\
{} & \textbf{Genetic} & \textbf{Hybrid VNS} & \textbf{Uniform} & \textbf{Multistart} & \textbf{Annealing} & \textbf{Tabu Search}\\
\hline
\textit{a280.tsp} & 2690.74 & 2678.60 & 2661.63 & 2695.19 & 2597.20 & 2653.86\\
\textit{bier127.tsp} & 118336.91 & 118639.82 & 119269.09 & 120322.98 & 118336.91 & 118722.26\\
\textit{ch130.tsp} & 6127.97 & 6153.70 & 6169.78 & 6232.01 & 6119.81 & 6452.45\\
\textit{ch150.tsp} & 6580.71 & 6608.86 & 6617.40 & 6630.18 & 6540.62 & 6939.93\\
\textit{d198.tsp} & 15935.33 & 15872.40 & 15941.29 & 16079.85 & 15898.05 & 15823.36\\
\textit{gil262.tsp} & 2454.80 & 2413.46 & 2449.23 & 2486.13 & 2391.00 & 2570.8\\
\textit{gr137.tsp} & 709.06 & 706.29 & 706.69 & 719.73 & 707.55 & 719.92\\
\textit{kroA150.tsp} & 26724.72 & 26715.75 & 26733.47 & 27004.19 & 26583.71 & 29872.56\\
\textit{kroA200.tsp} & 29621.32 & 29514.53 & 29654.05 & 29746.61 & 29470.83 & 29764.78\\
\textit{kroB150.tsp} & 26349.62 & 26243.64 & 26563.59 & 26906.92 & 26199.03 & 27428.81\\
\textit{kroB200.tsp} & 30024.80 & 29448.20 & 30012.64 & 30865.99 & 29487.73 & 33509.93\\
\textit{pr124.tsp} & 59074.80 & 59030.74 & 59030.74 & 59408.90 & 60088.84 & 60805.78\\
\textit{pr136.tsp} & 97551.40 & 96890.77 & 97919.03 & 99934.70 & 97108.59 & 104028.67\\
\textit{pr144.tsp} & 58568.77 & 58535.22 & 58535.22 & 58673.98 & 58761.43 & 58587.14\\
\textit{pr152.tsp} & 73687.11 & 73821.25 & 73844.13 & 74608.12 & 74022.66 & 75746.97\\
\textit{pr226.tsp} & 80644.46 & 80479.82 & 80612.15 & 80849.50 & 80570.65 & 84705.91\\
\textit{pr264.tsp} & 50217.59 & 50125.26 & 50712.35 & 51695.67 & 49203.39 & 51041.27\\
\textit{pr299.tsp} & 50274.10 & 49930.99 & 49572.62 & 50571.52 & 48667.16 & 51996.17\\
\textit{rd400.tsp} & 16118.40 & 15572.25 & 16025.16 & 16017.21 & 15610.31 & 16433.82\\
\textit{u159.tsp} & 42075.67 & 42075.67 & 42075.67 & 42874.46 & 42435.02 & 47110.63\\
\hline
\end{longtable}
}
{\footnotesize
\begin{longtable}[H]{lrrrrr}
\caption{Costo della soluzione ottenuta mediante algoritmi euristici con time limit di 10 minuti.}\\
\hline
{} & {} & \textbf{Hybrid VNS} & {} & \textbf{Simulated} & {}\\
{} & \textbf{Hybrid VNS} & \textbf{Uniform} & \textbf{Tabu Search} & \textbf{Annealing} & \textbf{Multistart}\\
\hline
\textit{d1291.tsp} & 55579.70 & 54622.05 & 54893.90 & 54309.32 & 54284.98\\
\textit{d1655.tsp} & 66661.25 & 66694.86 & 66361.42 & 65102.31 & 66410.23\\
\textit{d2103.tsp} & 82371.89 & 82225.44 & 82288.44 & 82037.22 & 82693.33\\
\textit{dsj1000.tsp} & 20252290.51 & 19767771.54 & 20290495.55 & 20552389.00 & 20016586.26\\
\textit{fl1400.tsp} & 21508.98 & 20932.21 & 21727.11 & 21306.06 & 21224.34\\
\textit{fl1577.tsp} & 24382.62 & 23287.37 & 24140.02 & 23875.58 & 23164.80\\
\textit{nrw1379.tsp} & 60399.89 & 60456.74 & 60205.58 & 59350.93 & 60114.27\\
\textit{pcb1173.tsp} & 61446.54 & 60073.98 & 60164.11 & 59879.89 & 61700.28\\
\textit{pr1002.tsp} & 275153.89 & 277315.21 & 270890.66 & 270671.89 & 272345.57\\
\textit{pr2392.tsp} & 408313.97 & 406410.65 & 409085.65 & 404233.03 & 405488.72\\
\textit{rl1304.tsp} & 282035.92 & 266352.94 & 278936.04 & 275343.16 & 274723.68\\
\textit{rl1323.tsp} & 289177.86 & 282450.23 & 290383.43 & 286016.11 & 289391.20\\
\textit{rl1889.tsp} & 345043.74 & 339914.33 & 346784.47 & 344552.56 & 340655.74\\
\textit{u1060.tsp} & 237825.17 & 236736.18 & 239277.18 & 233324.89 & 239326.80\\
\textit{u1432.tsp} & 164646.37 & 162328.27 & 162102.60 & 162103.55 & 165815.35\\
\textit{u1817.tsp} & 62460.55 & 61316.67 & 62030.58 & 62083.04 & 61739.76\\
\textit{u2152.tsp} & 70952.18 & 70145.91 & 70100.98 & 70103.33 & 69919.59\\
\textit{u2319.tsp} & 246272.89 & 246234.31 & 244854.41 & 244715.40 & 244783.90\\
\textit{vm1084.tsp} & 254733.26 & 257034.98 & 254005.22 & 251366.15 & 254167.66\\
\textit{vm1748.tsp} & 358335.39 & 357576.31 & 362593.77 & 359928.19 & 358742.92\\
\hline
\end{longtable}
}
%Bibliografia
\addcontentsline{toc}{chapter}{Bibliografia}
\bibliographystyle{plain}
\renewcommand{\bibname}{Bibliografia}
\bibliography{Chapters/biblio}
\end{document}