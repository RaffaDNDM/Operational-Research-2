\chapter{TSPlib}
Un'istanza di TSP viene definita normalmente da un grafo, per cui ad ogni nodo viene associata un numero intero (Ex. $\Pi = \{1,2,3,..,n\}$). \\
Una soluzione del problema è una sequenza di nodi che corrisponde ad una permutazione dell'istanza (es. $S = \{x_1,x_2,...,x_n\}$ tale che $x_i=x_j\;\;x_i \in \Pi\;\forall\;x_i\in S\;\wedge\; x_i!=x_j\;\forall\;i\neq j$). Poichè in questa variante non esiste alcuna origine, ogni tour può essere descritto da due versi di percorrenza e l'origine può essere un nodo qualsiasi del grafo.\\
La rappresentazione di tali istanze è stata svolta attraverso l'utilizzo del programma Gnuplot. Per avere dettagli riguardanti il suo utilizzo vedi Appendice \ref{gnuplot}.

Le istanze del problema, analizzate durante il corso, sono punti dello spazio 2D, identificati quindi da due coordinate ($x$,$y$).
Per generare istanza enormi del problema, si utilizza un approccio particolare in cui viene definito un insieme di punti a partire da un'immagine già esistente.\\
La vicinanza dei punti generati dipende dalla scala di grigi all'interno dell'immagine (es. generazione di punti a partire dal dipinto della Gioconda\cite{monnalisa}).\\
Le istanze che vengono elaborate dai programmi, creati durante il corso, utilizzano il template \textbf{TSPlib}. Di seguito viene riportato il contenuto di un file di questa tipologia.
 
\lstinputlisting[caption={\footnotesize{esempio.tsp}}, style=code, firstnumber=1, firstline=1, lastline=12, label=tsp_instance]{Source/esempio.tsp}

Le parole chiave più importanti, contenute in questi file \ref{tsp_instance}, sono:
\begin{itemize}
\item{\textbf{NAME}\\
seguito dal nome dell'istanza TSPlib}
\item{\textbf{COMMENT}\\
seguito da un commento associato all'istanza}
\item{\textbf{TYPE}\\
seguito dalla tipologia dell'istanza}
\item{\textbf{DIMENSION}\\
seguito dal numero di nodi nel grafo ($num\_nodi$)}
\item{\textbf{EDGE\_WEIGHT\_TYPE}\\
seguito dalla specifica del tipo di calcolo che viene effettuato per ricavare il costo del tour}
\item{\textbf{NODE\_COORD\_SECTION}\\
inizio della sezione composta di $num\_nodi$ righe in cui vengono riportate le caratteriste di ciascun nodo, nella forma seguente:
\begin{lstlisting}[linewidth=250pt,basicstyle=\footnotesize\sffamily,]     
indice_nodo  coordinata_x  coordinata_y
\end{lstlisting}}
\item{\textbf{EOF}\\
decreta la fine del file}
\end{itemize}

In alternativa alla sezione \textbf{NODE\_COORD\_SECTION} è possibile avere\\
\textbf{EDGE\_WEIGHT\_SECTION}, in cui sono riportati, all'interno di una matrice, i pesi di tutti gli archi. Questa tipologia di istanze non viene utilizzata dal programma da noi implementato.

%aggiungere edge weight section