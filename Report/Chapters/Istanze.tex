\chapter{Istanze del problema e soluzioni}
\section{Istanze}
Le istanze del problema solitamente sono punti dello spazio 2D, che sono quindi definiti da due coordinate, $x$ e $y$.
Per generare istanza enormi del problema, si utilizza un approccio particolare in cui viene definito un insieme di punti a partire da un'immagine già esistente.\\
La vicinanzae dei punti generati dipende dalla scala di grigi all'interno dell'immagine (Ex. generazione di punti a partire dal dipinto della Gioconda\cite{monnalisa}).\\
Le istanze che verranno utilizzate dai programmi, creati durante il corso, utilizzano il template \textbf{TSPlib}. Di seguito viene riportato il contenuto di un file di questa tipologia.
 
\lstinputlisting[caption={\footnotesize{esempio.tsp}}, style=code, firstnumber=1, firstline=1, lastline=12, label=tsp_instance]{Source/esempio.tsp}

Le parole chiave solitamente contenute in questi file \ref{tsp_instance} sono:
\begin{itemize}
\item{\textbf{NAME}\\
seguito dal nome dell'istanza TSPlib}
\item{\textbf{COMMENT}\\
seguito dal commento associato all'istanza}
\item{\textbf{TYPE}\\
seguito dalla tipologia dell'istanza}
\item{\textbf{DIMENSION}\\
seguito dal numero di nodi nel grafo ($num\_nodi$)}
\item{\textbf{EDGE\_WEIGHT\_TYPE}\\
seguito dalla specifica del tipo di calcolo che viene effettuato per ricavare il costo del tour}
\item{\textbf{NODE\_COORD\_SECTION}\\
inizio della sezione composta di $num\_nodi$ righe in cui vengono riportate le caratteriste di ciascun nodo, nella forma seguente:\\
\textbf{indice\_nodo  x  y}}
\item{\textbf{EOF}\\
decreta la fine del file}
\end{itemize}

\section{Soluzioni}

