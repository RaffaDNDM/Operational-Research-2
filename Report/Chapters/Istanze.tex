\chapter{Istanze del problema e soluzioni}
\section{Istanze}
Le istanze del problema solitamente sono punti dello spazio 2D, che sono quindi definiti da due coordinate, $x$ e $y$.
Per generare istanza enormi del problema, si utilizza un approccio particolare in cui viene definito un insieme di punti a partire da un'immagine già esistente.\\
La vicinanzae dei punti generati dipende dalla scala di grigi all'interno dell'immagine (Ex. generazione di punti a partire dal dipinto della Gioconda\cite{monnalisa}).\\
Le istanze che verranno utilizzate dai programmi, creati durante il corso, utilizzano il template \textbf{TSPlib}. Di seguito viene riportato il contenuto di un file di questa tipologia.
 
\lstinputlisting[caption={\footnotesize{esempio.tsp}}, style=code, firstnumber=1, firstline=1, lastline=12, label=tsp_instance]{Source/esempio.tsp}

Le parole chiave solitamente contenute in questi file \ref{tsp_instance} sono:
\begin{itemize}
\item{\textbf{NAME}\\
seguito dal nome dell'istanza TSPlib}
\item{\textbf{COMMENT}\\
seguito dal commento associato all'istanza}
\item{\textbf{TYPE}\\
seguito dalla tipologia dell'istanza}
\item{\textbf{DIMENSION}\\
seguito dal numero di nodi nel grafo ($num\_nodi$)}
\item{\textbf{EDGE\_WEIGHT\_TYPE}\\
seguito dalla specifica del tipo di calcolo che viene effettuato per ricavare il costo del tour}
\item{\textbf{NODE\_COORD\_SECTION}\\
inizio della sezione composta di $num\_nodi$ righe in cui vengono riportate le caratteriste di ciascun nodo, nella forma seguente:\\
\textbf{indice\_nodo  x  y}}
\item{\textbf{EOF}\\
decreta la fine del file}
\end{itemize}

\section{Soluzioni}
Una soluzione del problema è data da una sequenza di nodi, definita come una permutazione dell'istanza (Ex. $S = \{x_1,x_2,...,x_n\}$ tale che $x_i=x_j\;\;x_i \in \Pi\;\forall\;x_i\in S\;\wedge\; x_i!=x_j\;\forall\;i\neq j$). Poichè in questa variante non esiste alcuna origine, ogni tour può essere descritto da più permutazioni, due per ogni nodo del grafo. Una volta definita la soluzione $S$, infatti, questa può essere percorsa in entrambi i versi e l'origine può essere uno qualsiasi dei nodi del grafo.\\
\subsection{Gnuplot}
Una volta ottenuta la soluzione al problema di ottimizzazione ne viene disegnato il grafo per facilitarne la comprensione. Per fare ciò viene utilizzato un programma di tipo command-driven detto Gnuplot.\\
Per poter utilizzare Gnuplot all'interno del proprio programma esistono due metodi:\\
\begin{itemize}
\item{Collegarne la libreria ed invocare direttamente le sue funzioni all'interno del programma;}
\item{Collegare l'eseguibile interattivo al proprio programma. In questo caso i comandi deve essere passati all'eseguibile attraverso un file di testo.}
\end{itemize}
In questa trattazione è stato scelto il secondo metodo. All'interno del file è possibile specificare a Gnuplot le caratteristiche grafiche che deve aver il grafo (stile e spessore della linea, dimensione dei punti, se presenti delle etichette dove devono essere posizionate ecc). Di seguito è riportato un esempio di tale file.\\

\lstinputlisting[caption={\footnotesize{esempio.tsp}}, style=code, firstnumber=1, firstline=1, lastline=8, label=style_example]{Source/style_example.txt}

In questo caso "solution.dat" è il file che contiene le informazioni relative al grafico. In particolare sono presenti le coordinate dei punti, uno per riga, seguite dall'etichetta a loro associata.\\
\begin{table}[h]
\centering
\begin{tabular}{ccc} 
$x_i$         & $y_i$         & $label_i$
\end{tabular}
\end{table}
\\
La terza colonna può anche essere lasciata vuota, in quanto la presenza delle etichette non è obbligatoria (in quel caso deve essere modificato il file relativo allo stile rispetto all'esempio, eliminando questa specifica).
Nel caso in cui i nodi siano descritti su righe consecutive una all'altra (senza righe vuote tra loro) Gnuplot graficherà un ramo che colleghi i nodi su due righe adiacenti. Se invece si vogliono disegnare rami singolo è sufficiente inserire una riga vuota ogni due nodi (gli estremi del ramo che si vuole rappresentare).  






%creazione pipe