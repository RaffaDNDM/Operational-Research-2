\chapter{Introduzione}\label{intro}
La seguente trattazione analizza il Problema del Commesso Viaggiatore (Travelling Salesman Problem, TSP), che consiste nell'individuare un circuito hamiltoniano di costo minimo in un assegnato grafo orientato G=(V,A)\cite{TSP}. La formulazione matematica di tale problema è la seguente:
$$
x_{ij}=
\begin{cases}
1 & se\;l'arco\;(i,j)\in A\;viene\;scelto\;nel\;circuito\;ottimo\\
0 & altrimenti\\
\end{cases}
$$
\begin{align}
& min \underset{(i,j)\in A}\sum{c_{ij}\;x_{ij}} \\\notag \\
& \underset{(i,j)\in \delta^{-}(j)}\sum{\;x_{ij}} = 1 & \forall\;j\in V \\\notag \\
& \underset{(i,j)\in \delta^{+}(i)}\sum{\;x_{ij}} = 1 & \forall\;i\in V \\\notag \\
& \underset{(i,j)\in \delta^{+}(S)}\sum{\;x_{ij}} \geq 1 & S \subset V\; :\; 1\in S\\\notag \\
& x_{ij}\geq 0\;\; intero & (i,j)\in A.\\\notag
\end{align}
Tuttavia le soluzioni algoritmiche presentante risolvono una sua variante, detta simmetrica, che viene applicata ad un grafo completo non orientato G=(V,E).\\ Di seguito viene riportata la formulazione matematica di tale versione:\\
\begin{align}
& min \underset{e\in E}\sum{c_e\;x_e}\\
& \underset{e\in \delta(v)}\sum{\;x_e} = 2 & \forall\;v\in V \\
& \underset{e\in E(S)}\sum{\;x_e} \leq \vert S\vert - 1 & \forall\;S\underset{\neq} \subset V: \vert S\vert\geq 3.\\\notag
\end{align}
A livello commerciale esistono diverse tipologie di risolutori di problemi di programmazione lineare intera, basati sul Branch \& Bound. I più conosciuti in circolazione sono i seguenti:
\begin{itemize}
\item{\textbf{IBM ILOG CPLEX Optimization Studio}\cite{ILOG}\\
è un soluzione analitica, sviluppata dall'IBM e gratuita a livello accademico.}
\item{\textbf{FICO® Xpress Optimization}\cite{FICO}\\
è stato prodotto dalla Fair Isaac Corporation(FICO) ed è costituito da 4 componenti principali: FICO® Xpress Insight, FICO® Xpress Executor, FICO® Xpress Solver e FICO® Xpress Workbench. Questa soluzione è disponibile gratuitamente solo nella versione Community, in cui però vengono applicate restrizioni sul numero di righe e colonne del tableau, di token non lineari e di funzioni dell'utente.}
\item{\textbf{Gurobi}\cite{GUROBI}\\
è una soluzione, sviluppata dalla Gurobi Optimization, che viene rilasciata anche con una versione accademica.}
\item{\textbf{COIN Branch and Cut solver (CBC)}\cite{CBC}\\
è un risolutore MIP(mixed-integer program) open-source scritto in C++ e sviluppato dalla Computational Infrastructure for Operations Researc (COIN).}
\end{itemize}
Nel Capitolo \ref{CPLEX} vengono riportate diverse soluzioni math-euristiche e non per il problema del Commesso Viaggiatore, che fanno uso di ILOG CPLEX.\\
In commercio, il più noto ed efficiente software per la risoluzione del TSP è Concorde, sviluppato in ANSI C e disponibile per l'uso in ambito accademico\cite{concorde}.\\
Nel Capitolo \ref{HEURISTIC} vengono analizzati gli algoritmi euristici, sviluppati senza far uso di ILOG CPLEX.\\
Nel Capitolo \ref{PERF_PROF} vengono invece riportati i confronti, a livello temporale e di costo, delle soluzioni ottenute con i differenti algoritmi enunciati.\\
Nell'Appendice \ref{TSPlib}, \ref{CPLEX_func}, \ref{gnuplot}, \ref{perf_profile_py} vengono descritte rispettivamente la tipologia di istanze utilizzate, la documentazione utilizzata ed il funzionamento di CPLEX, il programma GNUPLOT utilizzato nella stampa delle soluzioni e il programma perfprof.py usato per creare i performance profile del Capitolo \ref{PERF_PROF}.\\
Tutti i tempi di esecuzione e i costi delle soluzioni, ottenuti mediante la fase di testing, sono consultabili invece nelle tabelle riportate nell'Appendice \ref{results}. Tutte le soluzioni descritte sono state implementate in linguaggio C e testate sul sistema operativo Windows 10 con Visual Studio, ed i sorgenti sono disponibili online\footnote{\url{https://github.com/RaffaDNDM/Operational-Research-2}}.