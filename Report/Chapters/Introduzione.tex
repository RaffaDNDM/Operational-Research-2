\chapter{Introduzione}
L'intera tesina verterà sul Travelling Salesman Problem. Quest'ultimo si pone l'obiettivo di trovare un tour ottimo,ovvero di costo minimo, all'intero di un grafo orientato.\\
In questa trattazione verranno analizzate soluzioni algoritmiche per una sua variante, detta simmetrica, che viene applicata a un grafo completo non orientato.\\ Di seguito viene riportata la formulazione matematica di tale versione:
$$
\begin{cases}
min \underset{e\in E}\sum{c_e\;x_e} \\\\
\underset{e\in \delta(v)}\sum{\;x_e} = 2\;\;\;\;\;\;\;\;\;\;\;\;\;\;\;\;\;\;\forall\;v\in V \\\\
\underset{e\in E(S)}\sum{\;x_e} \leq \vert S\vert - 1\;\;\;\;\;\;\;\;\forall\;S\underset{\neq} \subset V: \vert S\vert\geq 3\\
\end{cases}
$$
Un'istanza di tale problema viene definita normalmente da un grafo, per cui ad ogni nodo viene associata un numero intero (Ex. $\Pi = \{1,2,3,..,n\}$). 
Una soluzione del problema corrisponde invece ad una sequenza di nodi, definita come una permutazione dell'istanza (Ex. $S = \{x_1,x_2,...,x_n\}$ tale che $x_i=x_j\;\;x_i \in \Pi\;\forall\;x_i\in S\;\wedge\; x_i!=x_j\;\forall\;i\neq j$). Poichè in questa variante non esiste alcuna origine, ogni tour può essere descritto da più permutazioni, due per ogni nodo del grafo. Una volta definita la soluzione $S$, infatti, questa può essere percorsa in entrambi i versi e l'origine può essere uno qualsiasi dei nodi del grafo.\\
I risolutori che verranno applicati al problema sono di due tipologie:
\begin{itemize}
\item{\textbf{Risolutori esatti}\\ basati sul Branch \& Bound. I più conosciuti sono:
\begin{itemize}
\item{\textbf{IBM ILOG CPLEX}\\
gratuito se utilizzato solo a livello accademico.}
\item{\textbf{XPRESS}}
\item{\textbf{Gurobi}}
\item{\textbf{CBC}\\
l'unico Open-Source tra questi}
\end{itemize}
}
\item{\textbf{Risolutori euristici (meta-euristici)}\\
algoritmi che forniscono una soluzione approssimata.}
\end{itemize}
Esempio di risolutori: Concorde \cite{concorde}\\
William Cook \cite{cook}\\